\chapter{Наборы рукописных текстов}
\section{Латинские языки}

Подробное описание и сравнение наборов данных для распознавания текста можно посмотреть в \cite{hussain2015comprehensive} \url{http://www.iapr-tc11.org/mediawiki/index.php/Datasets_List}

\subsection{IAM}
Ссылка на базу - \url{http://www.iam.unibe.ch/~fkiwww/iamDB/data/}

Форма состоит из 4-х частей, которые разделяются прям линией. Первая часть содержит в себе заголовок и идетификационный номер формы. Вторая часть содержит машинопечатный текст, который в последствии пишется волонтером в третьей части формы. В четвертой части волонтер может оставить свое имя. 

\textbf{1. Этап. Выравнивание изображения.} Находим первую горизонтальную линию, высчитываем угол между первым слева и последним справа пикселем и на этот угол выравниваем скан формы.

\textbf{2. Этап. Сегментация на части.} Строим диаграму вертикальной проекции интенсивности пикселей и диаграмму самого длинного горизонтального пробега пикселя. Вычисляем порог для этих двух диграмм, который отделяет разделяющую вертикальную линию от всего остального. По порогу определяем точки сегментации и разделяем форму на 4 части.

 \textbf{3. Этап. Сегментация на строки.} Строим вертикальную проекцию интенсивности пикселей. Точки, в которых значение проекции равно нулю, помечаем как точки сегментации, вертикальные прямые через которые - разделяют строки рукописного текста. В местах, где проекция больше нуля, находим точки через которые можно провести прямую с наименьшим кол-вом пересечений со штрихами предыдущей и последующей строки. В местах пересечения таких прямых со связными компонентами возникает проблема отнесения компоненты к верхней или нижней строке, для решения этой проблемы используется следующий метод. Вычисляется центр масс компоненты и если он ближе к верхней строке, то компоненту относят к верхней строке, в противополножном случае - к нижней. Если центр масс лежит близко от разделяющей прямой, то компонента разделяется на две. 

 \textbf{4. Этап. Сегментация на слова.} Находим все связные компонены(СК) в строке. Вычисляем для каждой выпуклую оболочку и центр масс. Далее строим взвешенный неориентированный граф, вершинами которого являются СК, а ребрами - прямые, связывающие их центры масс, веса на ребрах равены длиннам отрезков, который соединяет выпуклые оболочки СК. По графу строим минимальное покрывающее дерево. Далее все веса из дерева разделяются на две группы: внутрисловестные и внесловестные. Разделение происходит по порогу, вычисленному методом Otzu.


\subsection{CEDAR}
\subsection{IRONFF}
\subsection{AWS}
\subsection{RIMES}
\subsection{CVL - Dataset}

\section{Кириллические языки}
\subsection{Кириллические наборы данных}

\url{https://www.reddit.com/r/MachineLearning/comments/2vvg9b/russian_script_training_set/}

\subsubsection{Национальный корпус русского языка}

\url{http://www.ruscorpora.ru/} - на этом сайте помещен корпус современного русского языка общим объемом более 600 млн слов. Корпус русского языка — это информационно-справочная система, основанная на собрании русских текстов в электронной форме.


\section{Synthetic Data}
more in \cite{bunke2003recognition} (3.2 Synthetic Training Data)

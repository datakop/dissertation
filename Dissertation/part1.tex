\chapter{Наборы рукописных текстов}
\section{Латинские языки}

Подробное описание и сравнение наборов данных для распознавания текста можно посмотреть в \cite{hussain2015comprehensive} \url{http://www.iapr-tc11.org/mediawiki/index.php/Datasets_List}

\subsection{IAM}
\subsection{CEDAR}
\subsection{IRONFF}
\subsection{AWS}
\subsection{RIMES}
\subsection{CVL - Dataset}

\section{Кириллические языки}
\subsection{Кириллические наборы данных}

\url{https://www.reddit.com/r/MachineLearning/comments/2vvg9b/russian_script_training_set/}

\subsubsection{Национальный корпус русского языка}

\url{http://www.ruscorpora.ru/} - на этом сайте помещен корпус современного русского языка общим объемом более 600 млн слов. Корпус русского языка — это информационно-справочная система, основанная на собрании русских текстов в электронной форме.


\section{Synthetic Data}
more in \cite{bunke2003recognition} (3.2 Synthetic Training Data)

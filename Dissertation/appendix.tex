\input{Dissertation/appendixsetup}   % Предварительные настройки для правильного подключения Приложений
\chapter{Программное обеспечение} \label{AppendixA}


\textbf{The Hidden Markov Model Toolkit (HTK)} is a portable toolkit for building and manipulating hidden Markov models. HTK is primarily used for speech recognition research although it has been used for numerous other applications including research into speech synthesis, character recognition and DNA sequencing. HTK is in use at hundreds of sites worldwide.

HTK consists of a set of library modules and tools available in C source form. The tools provide sophisticated facilities for speech analysis, HMM training, testing and results analysis. The software supports HMMs using both continuous density mixture Gaussians and discrete distributions and can be used to build complex HMM systems. The HTK release contains extensive documentation and examples. \cite{young2006htk}

\textbf{Torch} is a scientific computing framework with wide support for machine learning algorithms. It is easy to use and efficient, thanks to an easy and fast scripting language, LuaJIT, and an underlying C/CUDA implementation.

\noindent A summary of core features:
\begin{itemize}
  \item a powerful N-dimensional array
  \item lots of routines for indexing, slicing, transposing, ...
  \item amazing interface to C, via LuaJIT
  \item linear algebra routines
  \item neural network, and energy-based models
  \item numeric optimization routines
  \item Fast and efficient GPU support
  \item Embeddable, with ports to iOS, Android and FPGA backends
\end{itemize}

\textbf{Lasagne} is a lightweight library to build and train neural networks in Theano.

\textbf{Theano} is a Python library that allows you to define, optimize, and evaluate mathematical expressions involving multi-dimensional arrays efficiently.


\textbf{Handwritten Character Recognition System using Neural Networks} developed using MATLAB Neural Network and Image Processing tool box. This system has been developed using existing algorithms like Preprocessing and Feature Extraction techniques.

For More Info http://www.slideshare.net/chiranjeeviadi/hand-written-character-recognition-using-neural-networks

https://github.com/sachinkariyattin/HWCR


\textbf{CTC implementation} https://github.com/daweileng/Precise-CTC.




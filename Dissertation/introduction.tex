\chapter*{Введение}                         % Заголовок
\addcontentsline{toc}{chapter}{Введение}    % Добавляем его в оглавление



Человек использует письменность, письменные нотации и иллюстрации для передачи и хранения информации. Письменность представляет собой знаковую систему, предназначенную для формализации, фиксации, сохранения и передачи информации. Продуктом письменности является текст, текст может храниться в нескольких форматах: на бумаге в печатном или рукописном виде, в электронном формате в определенной кодировке(ASCII, UTF\cite{unicode1991unicode} и т.д.).

Иллюстрация — рисунок, фотография, гравюра или другое изображение, поясняющее текст. Продуктом иллюстрации является изображение, которое может хранится как на бумаге, как в электронном формате(JPEG, PNG..), так и может быть описано на языке разметки(SVG).

Нотация - система условных обозначений, принятая в какой-либо области знаний или деятельности. Включает множество символов, используемых для представления понятий и их взаимоотношений, составляющее алфавит нотации, а также правила их применения.

Основной задачей распознавания образов (pattern recognition) является воссоздание человеческого восприятия в искусственных интеллектуальных системах. Особым видом такого восприятия является визуальное восприятие, возможность чтения и интерпретации текста, нотаций, иллюстраций. На данных момент не существует систем, которые могли бы с человеческой точностью решать эти задачи. Такие системы называются Reading Systems, исследования которых контролируется специальным Техническим Коммитетом TC-11, который входит в International Association for Pattern Recognition.

Европейские проекты по распознаванию исторических рукописей \cite{vidal2014d5, romero2012multimodal}.

Диссертации на схожую тему \cite{guillevic1995unconstrained, vinciarelli2003offline, tay2002offline, fischer2012handwriting, romero2012multimodal}
\chapter{Архитектура HWR систем} \label{chapt4}

Литература \cite{cheriet2007character, romero2012multimodal, bunke2007off}

\section{Проблемы с которыми приходится сталкиваться:} \label{sect4_1}

Зашумленное изображение
Выделение текста на изображении
Сегментация текста на строки
Сегментация строк на слова
Вариативность почерка
Язык рукописи
Размерность словаря
Словарь может быть маленький
Может быть большой
А может и не быть заданного словаря, тогда говорят о задаче без словарного распознавания 
Сегментация слова на примитивы
Проблема сегментации возникает на всех уровнях СРРТ, начиная от сегментации текста от фона изображения, заканчивая восстановлением сегментации для уже распознанных слов в целях выявления ошибок.
Распознавание примитивов из заданного класса и отбраковка остальных
Распознавание слова
Построение модели языка


\section{Подготовка изображения к Распознаванию} \label{sect4_2}

\subsection{Предобработка скана изображения} \label{subsect4_2_1}

\cite{durebrandt2015segmentation}

Удаление шума, применение методов threshoulding-а
Выделение области интереса


\subsection{Предобработка изображения текста} \label{subsect4_2_2}

Цель этапа предобработки строк - применить последовательность процедур нормализации к изображениям строк с целью сокращения вариативности внешней стилистики рукописного текста. Некоторые из возможных процедур нормализации описаны в следующих разделах.

\textbf{Скелетонизация (thinning or skeletonization).} Целью данной процедуры является нормализация ширины каждой линии на изображении к ширине 1-ого пикселя, сохраняя при это топологию структуры слова. \cite{suen1994thinning}

\textbf{Коррекция базовой линии(baseline correction or skew correction).} computing a baseline estimate by interpolating local contour minima (cf. [...]) works well for Roman script, but will be likely to fail for Arabic. \cite{plotz2009markov}

More widely applicable is the implicit estimation of the baseline orientation by maximizing the entropy of the horizontal projection histogram (cf., e.g., [...]). \cite{plotz2009markov}

\textbf{slant angle correction} The slant angle, i.e., the tilt of individual handwritten characters with respect to the vertical, is compensated by applying a shear transform to the text-line image. \cite{plotz2009markov}

\textbf{size of the handwriting.} A quite robust method for normalizing the size of Roman script was proposed in \cite{madhvanath1999chaincode} which uses an estimate of average character width.

Более подробно про текхники предобработки текстового изображения можно почитать в \cite{fujisawa2007robustness}.

\subsection{Сегментация строки и слова} \label{subsect4_2_3}

A good overviewof algorithms for text-line extraction is given in \cite{likforman2007text}, \cite{durebrandt2015segmentation}.
\textbf{Препроцессинг и нормализация строк.} \cite{durebrandt2015segmentation}

\textbf{Сегментация строк на слова.} \cite{durebrandt2015segmentation}

\textbf{Нормализация изображений слов.} \cite{durebrandt2015segmentation}

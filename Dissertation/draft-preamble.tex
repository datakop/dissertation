%\RequirePackage[l2tabu,orthodox]{nag} % Раскомментировав, можно в логе получать рекомендации относительно правильного использования пакетов и предупреждения об устаревших и нерекомендуемых пакетах
% Формат А4, 14pt (ГОСТ Р 7.0.11-2011, 5.3.6)
\documentclass[a4paper,14pt]{extreport}

%%%%%%%%%%%%%%%%%%%%%%%%%%%%%%%%%%%%%%%%%%%%%%%%%%%%%%%%%%%%%%%%%%%%%%%%%%%
%%%% Проверка используемого TeX-движка %%%
\usepackage{iftex}
\newif\ifxetexorluatex   % определяем новый условный оператор (http://tex.stackexchange.com/a/47579/79756)
\ifXeTeX
    \xetexorluatextrue
\else
    \ifLuaTeX
        \xetexorluatextrue
    \else
        \xetexorluatexfalse
    \fi
\fi

%%% Поля и разметка страницы %%%
\usepackage{pdflscape}                              % Для включения альбомных страниц
\usepackage{geometry}                               % Для последующего задания полей

%%% Математические пакеты %%%
\usepackage{amsthm,amsfonts,amsmath,amssymb,amscd}  % Математические дополнения от AMS
\usepackage{mathtools}                              % Добавляет окружение multlined

%%%% Установки для размера шрифта 14 pt %%%%
%% Формирование переменных и констант для сравнения (один раз для всех подключаемых файлов)%%
%% должно располагаться до вызова пакета fontspec или polyglossia, потому что они сбивают его работу
\newlength{\curtextsize}
\newlength{\bigtextsize}
\setlength{\bigtextsize}{13.9pt}

\makeatletter
%\show\f@size                                       % неплохо для отслеживания, но вызывает стопорение процесса, если документ компилируется без команды  -interaction=nonstopmode 
\setlength{\curtextsize}{\f@size pt}
\makeatother

%%% Кодировки и шрифты %%%
\ifxetexorluatex
    \usepackage{polyglossia}                        % Поддержка многоязычности (fontspec подгружается автоматически)
\else
    \RequirePDFTeX                                  % tests for PDFTEX use and throws an error if a different engine is being used
   %%% Решение проблемы копирования текста в буфер кракозябрами
%    \input glyphtounicode.tex
%    \input glyphtounicode-cmr.tex %from pdfx package
%    \pdfgentounicode=1
    \usepackage{cmap}                               % Улучшенный поиск русских слов в полученном pdf-файле
    \defaulthyphenchar=127                          % Если стоит до fontenc, то переносы не впишутся в выделяемый текст при копировании его в буфер обмена
    \usepackage[T2A]{fontenc}                       % Поддержка русских букв
    \usepackage[utf8]{inputenc}                     % Кодировка utf8
    \usepackage[english, russian]{babel}            % Языки: русский, английский
    \IfFileExists{pscyr.sty}{\usepackage{pscyr}}{}  % Красивые русские шрифты
\fi

%%% Оформление абзацев %%%
\usepackage{indentfirst}                            % Красная строка

%%% Цвета %%%
\usepackage[dvipsnames,usenames]{color}
\usepackage{colortbl}
%\usepackage[dvipsnames, table, hyperref, cmyk]{xcolor} % Вероятно, более новый вариант, вместо предыдущих двух строк. Конвертация всех цветов в cmyk заложена как удовлетворение возможного требования типографий. Возможно конвертирование и в rgb.

%%% Таблицы %%%
\usepackage{longtable}                              % Длинные таблицы
\usepackage{multirow,makecell,array}                % Улучшенное форматирование таблиц
\usepackage{booktabs}                               % Возможность оформления таблиц в классическом книжном стиле (при правильном использовании не противоречит ГОСТ)

%%% Общее форматирование
\usepackage{soulutf8}                               % Поддержка переносоустойчивых подчёркиваний и зачёркиваний
\usepackage{icomma}                                 % Запятая в десятичных дробях


%%% Гиперссылки %%%
\usepackage{hyperref}

%%% Изображения %%%
\usepackage{graphicx}                               % Подключаем пакет работы с графикой

%%% Списки %%%
\usepackage{enumitem}

%%% Подписи %%%
\usepackage{caption}                                % Для управления подписями (рисунков и таблиц) % Может управлять номерами рисунков и таблиц с caption %Иногда может управлять заголовками в списках рисунков и таблиц
\usepackage{subcaption}                             % Работа с подрисунками и подобным

%%% Интервалы %%%
\usepackage[onehalfspacing]{setspace}               % Опция запуска пакета правит не только интервалы в обычном тексте, но и формульные

%%% Счётчики %%%
\usepackage[figure,table]{totalcount}               % Счётчик рисунков и таблиц
\usepackage{totcount}                               % Пакет создания счётчиков на основе последнего номера подсчитываемого элемента (может требовать дважды компилировать документ)
\usepackage{totpages}                               % Счётчик страниц, совместимый с hyperref (ссылается на номер последней страницы). Желательно ставить последним пакетом в преамбуле

%%% Продвинутое управление групповыми ссылками (пока только формулами) %%%
\ifxetexorluatex
    \usepackage{cleveref}                           % cleveref корректно считывает язык из настроек polyglossia
\else
    \usepackage[russian]{cleveref}                  % cleveref имеет сложности со считыванием языка из babel. Такое решение русификации вывода выбрано вместо определения в documentclass из опасности что-то лишнее передать во все остальные пакеты, включая библиографию.
\fi
\creflabelformat{equation}{#2#1#3}                  % Формат по умолчанию ставил круглые скобки вокруг каждого номера ссылки, теперь просто номера ссылок без какого-либо дополнительного оформления

  % Пакеты общие для диссертации и автореферата
%%% Проверка используемого TeX-движка %%%
\usepackage{iftex}
\newif\ifxetexorluatex   % определяем новый условный оператор (http://tex.stackexchange.com/a/47579/79756)
\ifXeTeX
    \xetexorluatextrue
\else
    \ifLuaTeX
        \xetexorluatextrue
    \else
        \xetexorluatexfalse
    \fi
\fi

%%% Поля и разметка страницы %%%
\usepackage{pdflscape}                              % Для включения альбомных страниц
\usepackage{geometry}                               % Для последующего задания полей

%%% Математические пакеты %%%
\usepackage{amsthm,amsfonts,amsmath,amssymb,amscd}  % Математические дополнения от AMS
\usepackage{mathtools}                              % Добавляет окружение multlined

%%%% Установки для размера шрифта 14 pt %%%%
%% Формирование переменных и констант для сравнения (один раз для всех подключаемых файлов)%%
%% должно располагаться до вызова пакета fontspec или polyglossia, потому что они сбивают его работу
\newlength{\curtextsize}
\newlength{\bigtextsize}
\setlength{\bigtextsize}{13.9pt}

\makeatletter
%\show\f@size                                       % неплохо для отслеживания, но вызывает стопорение процесса, если документ компилируется без команды  -interaction=nonstopmode 
\setlength{\curtextsize}{\f@size pt}
\makeatother

%%% Кодировки и шрифты %%%
\ifxetexorluatex
    \usepackage{polyglossia}                        % Поддержка многоязычности (fontspec подгружается автоматически)
\else
    \RequirePDFTeX                                  % tests for PDFTEX use and throws an error if a different engine is being used
   %%% Решение проблемы копирования текста в буфер кракозябрами
%    \input glyphtounicode.tex
%    \input glyphtounicode-cmr.tex %from pdfx package
%    \pdfgentounicode=1
    \usepackage{cmap}                               % Улучшенный поиск русских слов в полученном pdf-файле
    \defaulthyphenchar=127                          % Если стоит до fontenc, то переносы не впишутся в выделяемый текст при копировании его в буфер обмена
    \usepackage[T2A]{fontenc}                       % Поддержка русских букв
    \usepackage[utf8]{inputenc}                     % Кодировка utf8
    \usepackage[english, russian]{babel}            % Языки: русский, английский
    \IfFileExists{pscyr.sty}{\usepackage{pscyr}}{}  % Красивые русские шрифты
\fi

%%% Оформление абзацев %%%
\usepackage{indentfirst}                            % Красная строка

%%% Цвета %%%
\usepackage[dvipsnames,usenames]{color}
\usepackage{colortbl}
%\usepackage[dvipsnames, table, hyperref, cmyk]{xcolor} % Вероятно, более новый вариант, вместо предыдущих двух строк. Конвертация всех цветов в cmyk заложена как удовлетворение возможного требования типографий. Возможно конвертирование и в rgb.

%%% Таблицы %%%
\usepackage{longtable}                              % Длинные таблицы
\usepackage{multirow,makecell,array}                % Улучшенное форматирование таблиц
\usepackage{booktabs}                               % Возможность оформления таблиц в классическом книжном стиле (при правильном использовании не противоречит ГОСТ)

%%% Общее форматирование
\usepackage{soulutf8}                               % Поддержка переносоустойчивых подчёркиваний и зачёркиваний
\usepackage{icomma}                                 % Запятая в десятичных дробях


%%% Гиперссылки %%%
\usepackage{hyperref}

%%% Изображения %%%
\usepackage{graphicx}                               % Подключаем пакет работы с графикой

%%% Списки %%%
\usepackage{enumitem}

%%% Подписи %%%
\usepackage{caption}                                % Для управления подписями (рисунков и таблиц) % Может управлять номерами рисунков и таблиц с caption %Иногда может управлять заголовками в списках рисунков и таблиц
\usepackage{subcaption}                             % Работа с подрисунками и подобным

%%% Интервалы %%%
\usepackage[onehalfspacing]{setspace}               % Опция запуска пакета правит не только интервалы в обычном тексте, но и формульные

%%% Счётчики %%%
\usepackage[figure,table]{totalcount}               % Счётчик рисунков и таблиц
\usepackage{totcount}                               % Пакет создания счётчиков на основе последнего номера подсчитываемого элемента (может требовать дважды компилировать документ)
\usepackage{totpages}                               % Счётчик страниц, совместимый с hyperref (ссылается на номер последней страницы). Желательно ставить последним пакетом в преамбуле

%%% Продвинутое управление групповыми ссылками (пока только формулами) %%%
\ifxetexorluatex
    \usepackage{cleveref}                           % cleveref корректно считывает язык из настроек polyglossia
\else
    \usepackage[russian]{cleveref}                  % cleveref имеет сложности со считыванием языка из babel. Такое решение русификации вывода выбрано вместо определения в documentclass из опасности что-то лишнее передать во все остальные пакеты, включая библиографию.
\fi
\creflabelformat{equation}{#2#1#3}                  % Формат по умолчанию ставил круглые скобки вокруг каждого номера ссылки, теперь просто номера ссылок без какого-либо дополнительного оформления

%%%%%%%%%%%%%%%%%%%%%%%%%%%%%%%%%%%%%%%%%%%%%%%%%%%%%%%%%%%%%%%%%%%%%%%%%%%
%%%%%%%%%%%%%%%%%%%%%%%%%%%%%%%%%%%%%%%%%%%%%%%%%%%%%%%%%%%%%%%%%%%%%%%%%%%
%%%% Колонтитулы %%%
\usepackage{fancyhdr}

%%% Прикладные пакеты %%% 
\usepackage{calc}               % Пакет для расчётов параметров, например длины
%\usepackage{etoolbox}          % ради функции patchcmd для управления списком литературы

\usepackage {interfaces-base}   % Набор базовых интерфейсов к некоторым пакетам, конкретные реализации загружаются в стиле

%%% Заголовки %%%
\usepackage{titlesec}           % Пакет настройки шрифтов заголовков в тексте

%%% Оглавление %%%
\usepackage{tocloft}

%%% Счётчики %%%
\usepackage{chngcntr}           % оперативная перенастройка счётчиков         % Пакеты для диссертации
%%% Колонтитулы %%%
\usepackage{fancyhdr}

%%% Прикладные пакеты %%% 
\usepackage{calc}               % Пакет для расчётов параметров, например длины
%\usepackage{etoolbox}          % ради функции patchcmd для управления списком литературы

\usepackage {interfaces-base}   % Набор базовых интерфейсов к некоторым пакетам, конкретные реализации загружаются в стиле

%%% Заголовки %%%
\usepackage{titlesec}           % Пакет настройки шрифтов заголовков в тексте

%%% Оглавление %%%
\usepackage{tocloft}

%%% Счётчики %%%
\usepackage{chngcntr}           % оперативная перенастройка счётчиков
%%%%%%%%%%%%%%%%%%%%%%%%%%%%%%%%%%%%%%%%%%%%%%%%%%%%%%%%%%%%%%%%%%%%%%%%%%%
%%%%%%%%%%%%%%%%%%%%%%%%%%%%%%%%%%%%%%%%%%%%%%%%%%%%%%%%%%%%%%%%%%%%%%%%%%%
%\usepackage{tabularx,tabulary}  %таблицы с автоматически подбирающейся шириной столбцов

% Листинги с исходным кодом программ
\usepackage{fancyvrb}
\usepackage{listings}

% Плавающие окружения. во многом лучше пакета float
\usepackage{floatrow}

% Русская традиция начертания греческих букв
%\usepackage{upgreek} % прямые греческие ради русской традиции        % Пакеты для специфических пользовательских задач
\usepackage{tabularx,tabulary}  %таблицы с автоматически подбирающейся шириной столбцов

% Листинги с исходным кодом программ
\usepackage{fancyvrb}
\usepackage{listings}

% Плавающие окружения. во многом лучше пакета float
\usepackage{floatrow}

% Русская традиция начертания греческих букв
%\usepackage{upgreek} % прямые греческие ради русской традиции

%%%%%%%%%%%%%%%%%%%%%%%%%%%%%%%%%%%%%%%%%%%%%%%%%%%%%%%%%%%%%%%%%%%%%%%%%%%
%%%%%%%%%%%%%%%%%%%%%%%%%%%%%%%%%%%%%%%%%%%%%%%%%%%%%%%%%%%%%%%%%%%%%%%%%%%
%%%%%%%%%%%%%%%%%%%%%%%%%%%%%%%%%%%%%%%%%%%%%%%%%%%%%%%
%%%% Файл упрощённых настроек шаблона диссертации %%%%
%%%%%%%%%%%%%%%%%%%%%%%%%%%%%%%%%%%%%%%%%%%%%%%%%%%%%%

%%%        Подключение пакетов                 %%%
\usepackage{ifthen}                 % добавляет ifthenelse
%%% Инициализирование переменных, не трогать!  %%%
\newcounter{intvl}
\newcounter{otstup}
\newcounter{contnumeq}
\newcounter{contnumfig}
\newcounter{contnumtab}
\newcounter{pgnum}
\newcounter{bibliosel}
\newcounter{chapstyle}
\newcounter{headingdelim}
\newcounter{headingalign}
\newcounter{headingsize}
\newcounter{tabcap}
\newcounter{tablaba}
\newcounter{tabtita}
%%%%%%%%%%%%%%%%%%%%%%%%%%%%%%%%%%%%%%%%%%%%%%%%%%

%%% Область упрощённого управления оформлением %%%

%% Интервал между заголовками и между заголовком и текстом
% Заголовки отделяют от текста сверху и снизу тремя интервалами (ГОСТ Р 7.0.11-2011, 5.3.5)
\setcounter{intvl}{3}               % Коэффициент кратности к размеру шрифта

%% Отступы у заголовков в тексте
\setcounter{otstup}{0}              % 0 --- без отступа; 1 --- абзацный отступ

%% Нумерация формул, таблиц и рисунков
\setcounter{contnumeq}{0}           % Нумерация формул: 0 --- пораздельно (во введении подряд, без номера раздела); 1 --- сквозная нумерация по всей диссертации
\setcounter{contnumfig}{0}          % Нумерация рисунков: 0 --- пораздельно (во введении подряд, без номера раздела); 1 --- сквозная нумерация по всей диссертации
\setcounter{contnumtab}{1}          % Нумерация таблиц: 0 --- пораздельно (во введении подряд, без номера раздела); 1 --- сквозная нумерация по всей диссертации

%% Оглавление
\setcounter{pgnum}{1}               % 0 --- номера страниц никак не обозначены; 1 --- Стр. над номерами страниц (дважды компилировать после изменения)

%% Библиография
\setcounter{bibliosel}{1}           % 0 --- встроенная реализация с загрузкой файла через движок bibtex8; 1 --- реализация пакетом biblatex через движок biber

%% Текст и форматирование заголовков
\setcounter{chapstyle}{1}           % 0 --- разделы только под номером; 1 --- разделы с названием "Глава" перед номером
\setcounter{headingdelim}{1}        % 0 --- номер отделен пропуском в 1em или \quad; 1 --- номера разделов и приложений отделены точкой с пробелом, подразделы пропуском без точки; 2 --- номера разделов, подразделов и приложений отделены точкой с пробелом.

%% Выравнивание заголовков в тексте
\setcounter{headingalign}{0}        % 0 --- по центру; 1 --- по левому краю

%% Размеры заголовков в тексте
\setcounter{headingsize}{0}         % 0 --- по ГОСТ, все всегда 14 пт; 1 --- пропорционально изменяющийся размер в зависимости от базового шрифта

%% Подпись таблиц
\setcounter{tabcap}{0}              % 0 --- по ГОСТ, номер таблицы и название разделены тире, выровнены по левому краю, при необходимости на нескольких строках; 1 --- подпись таблицы не по ГОСТ, на двух и более строках, дальнейшие настройки: 
%Выравнивание первой строки, с подписью и номером
\setcounter{tablaba}{2}             % 0 --- по левому краю; 1 --- по центру; 2 --- по правому краю
%Выравнивание строк с самим названием таблицы
\setcounter{tabtita}{1}             % 0 --- по левому краю; 1 --- по центру; 2 --- по правому краю

%%% Цвета гиперссылок %%%
% Latex color definitions: http://latexcolor.com/
\definecolor{linkcolor}{rgb}{0.9,0,0}
\definecolor{citecolor}{rgb}{0,0.6,0}
\definecolor{urlcolor}{rgb}{0,0,1}
%\definecolor{linkcolor}{rgb}{0,0,0} %black
%\definecolor{citecolor}{rgb}{0,0,0} %black
%\definecolor{urlcolor}{rgb}{0,0,0} %black               % Упрощённые настройки шаблона
%%%%%%%%%%%%%%%%%%%%%%%%%%%%%%%%%%%%%%%%%%%%%%%%%%%%%%
%%%% Файл упрощённых настроек шаблона диссертации %%%%
%%%%%%%%%%%%%%%%%%%%%%%%%%%%%%%%%%%%%%%%%%%%%%%%%%%%%%

%%%        Подключение пакетов                 %%%
\usepackage{ifthen}                 % добавляет ifthenelse
%%% Инициализирование переменных, не трогать!  %%%
\newcounter{intvl}
\newcounter{otstup}
\newcounter{contnumeq}
\newcounter{contnumfig}
\newcounter{contnumtab}
\newcounter{pgnum}
\newcounter{bibliosel}
\newcounter{chapstyle}
\newcounter{headingdelim}
\newcounter{headingalign}
\newcounter{headingsize}
\newcounter{tabcap}
\newcounter{tablaba}
\newcounter{tabtita}
%%%%%%%%%%%%%%%%%%%%%%%%%%%%%%%%%%%%%%%%%%%%%%%%%%

%%% Область упрощённого управления оформлением %%%

%% Интервал между заголовками и между заголовком и текстом
% Заголовки отделяют от текста сверху и снизу тремя интервалами (ГОСТ Р 7.0.11-2011, 5.3.5)
\setcounter{intvl}{3}               % Коэффициент кратности к размеру шрифта

%% Отступы у заголовков в тексте
\setcounter{otstup}{0}              % 0 --- без отступа; 1 --- абзацный отступ

%% Нумерация формул, таблиц и рисунков
\setcounter{contnumeq}{0}           % Нумерация формул: 0 --- пораздельно (во введении подряд, без номера раздела); 1 --- сквозная нумерация по всей диссертации
\setcounter{contnumfig}{0}          % Нумерация рисунков: 0 --- пораздельно (во введении подряд, без номера раздела); 1 --- сквозная нумерация по всей диссертации
\setcounter{contnumtab}{1}          % Нумерация таблиц: 0 --- пораздельно (во введении подряд, без номера раздела); 1 --- сквозная нумерация по всей диссертации

%% Оглавление
\setcounter{pgnum}{1}               % 0 --- номера страниц никак не обозначены; 1 --- Стр. над номерами страниц (дважды компилировать после изменения)

%% Библиография
\setcounter{bibliosel}{1}           % 0 --- встроенная реализация с загрузкой файла через движок bibtex8; 1 --- реализация пакетом biblatex через движок biber

%% Текст и форматирование заголовков
\setcounter{chapstyle}{1}           % 0 --- разделы только под номером; 1 --- разделы с названием "Глава" перед номером
\setcounter{headingdelim}{1}        % 0 --- номер отделен пропуском в 1em или \quad; 1 --- номера разделов и приложений отделены точкой с пробелом, подразделы пропуском без точки; 2 --- номера разделов, подразделов и приложений отделены точкой с пробелом.

%% Выравнивание заголовков в тексте
\setcounter{headingalign}{0}        % 0 --- по центру; 1 --- по левому краю

%% Размеры заголовков в тексте
\setcounter{headingsize}{0}         % 0 --- по ГОСТ, все всегда 14 пт; 1 --- пропорционально изменяющийся размер в зависимости от базового шрифта

%% Подпись таблиц
\setcounter{tabcap}{0}              % 0 --- по ГОСТ, номер таблицы и название разделены тире, выровнены по левому краю, при необходимости на нескольких строках; 1 --- подпись таблицы не по ГОСТ, на двух и более строках, дальнейшие настройки: 
%Выравнивание первой строки, с подписью и номером
\setcounter{tablaba}{2}             % 0 --- по левому краю; 1 --- по центру; 2 --- по правому краю
%Выравнивание строк с самим названием таблицы
\setcounter{tabtita}{1}             % 0 --- по левому краю; 1 --- по центру; 2 --- по правому краю

%%% Цвета гиперссылок %%%
% Latex color definitions: http://latexcolor.com/
\definecolor{linkcolor}{rgb}{0.9,0,0}
\definecolor{citecolor}{rgb}{0,0.6,0}
\definecolor{urlcolor}{rgb}{0,0,1}
%\definecolor{linkcolor}{rgb}{0,0,0} %black
%\definecolor{citecolor}{rgb}{0,0,0} %black
%\definecolor{urlcolor}{rgb}{0,0,0} %black

%%%%%%%%%%%%%%%%%%%%%%%%%%%%%%%%%%%%%%%%%%%%%%%%%%%%%%%%%%%%%%%%%%%%%%%%%%%
%%%%%%%%%%%%%%%%%%%%%%%%%%%%%%%%%%%%%%%%%%%%%%%%%%%%%%%%%%%%%%%%%%%%%%%%%%%
%%%% Переопределение именований, чтобы можно было и в преамбуле использовать %%%
\renewcommand{\chaptername}{Глава}
\renewcommand{\appendixname}{Приложение} % (ГОСТ Р 7.0.11-2011, 5.7)
       % Переопределение именований, чтобы можно было и в преамбуле использовать
%%% Переопределение именований, чтобы можно было и в преамбуле использовать %%%
\renewcommand{\chaptername}{Глава}
\renewcommand{\appendixname}{Приложение} % (ГОСТ Р 7.0.11-2011, 5.7)

%%%%%%%%%%%%%%%%%%%%%%%%%%%%%%%%%%%%%%%%%%%%%%%%%%%%%%%%%%%%%%%%%%%%%%%%%%%
%%%%%%%%%%%%%%%%%%%%%%%%%%%%%%%%%%%%%%%%%%%%%%%%%%%%%%%%%%%%%%%%%%%%%%%%%%%
%% Новые переменные, которые могут использоваться во всём проекте
\newcommand{\authorbibtitle}{Публикации автора по теме диссертации}
\newcommand{\fullbibtitle}{Список литературы} % (ГОСТ Р 7.0.11-2011, 4)
  % Новые переменные, которые могут использоваться во всём проекте
% Новые переменные, которые могут использоваться во всём проекте
\newcommand{\authorbibtitle}{Публикации автора по теме диссертации}
\newcommand{\fullbibtitle}{Список литературы} % (ГОСТ Р 7.0.11-2011, 4)

%%%%%%%%%%%%%%%%%%%%%%%%%%%%%%%%%%%%%%%%%%%%%%%%%%%%%%%%%%%%%%%%%%%%%%%%%%%
%%%%%%%%%%%%%%%%%%%%%%%%%%%%%%%%%%%%%%%%%%%%%%%%%%%%%%%%%%%%%%%%%%%%%%%%%%%
%%% Основные сведения %%%
\newcommand{\thesisAuthor}             % Диссертация, ФИО автора
{%
    \texorpdfstring{% \texorpdfstring takes two arguments and uses the first for (La)TeX and the second for pdf
        \todo{Фамилия Имя Отчество автора}% так будет отображаться на титульном листе или в тексте, где будет использоваться переменная
    }{%
        Фамилия, Имя Отчество% эта запись для свойств pdf-файла. В таком виде, если pdf будет обработан программами для сбора библиографических сведений, будет правильно представлена фамилия.
    }%
}
\newcommand{\thesisUdk}                % Диссертация, УДК
{\todo{xxx.xxx}}
\newcommand{\thesisTitle}              % Диссертация, название
{\texorpdfstring{\todo{\MakeUppercase{Название диссертационной работы}}}{Название диссертационной работы}}
\newcommand{\thesisSpecialtyNumber}    % Диссертация, специальность, номер
{\texorpdfstring{\todo{XX.XX.XX}}{XX.XX.XX}}
\newcommand{\thesisSpecialtyTitle}     % Диссертация, специальность, название
{\texorpdfstring{\todo{Название специальности}}{Название специальности}}
\newcommand{\thesisDegree}             % Диссертация, научная степень
{\todo{кандидата физико-математических наук}}
\newcommand{\thesisCity}               % Диссертация, город защиты
{\todo{Город}}
\newcommand{\thesisYear}               % Диссертация, год защиты
{\todo{20XX}}
\newcommand{\thesisOrganization}       % Диссертация, организация
{\todo{Название учреждения, в~котором выполнялась данная диссертационная работа}}

\newcommand{\thesisInOrganization}       % Диссертация, организация в предложном падеже: Работа выполнена в ...
{\todo{учреждении, в~котором выполнялась данная диссертационная работа}}

\newcommand{\supervisorFio}            % Научный руководитель, ФИО
{\todo{Фамилия Имя Отчество}}
\newcommand{\supervisorRegalia}        % Научный руководитель, регалии
{\todo{уч. степень, уч. звание}}

\newcommand{\opponentOneFio}           % Оппонент 1, ФИО
{\todo{Фамилия Имя Отчество}}
\newcommand{\opponentOneRegalia}       % Оппонент 1, регалии
{\todo{доктор физико-математических наук, профессор}}
\newcommand{\opponentOneJobPlace}      % Оппонент 1, место работы
{\todo{Не очень длинное название для места работы}}
\newcommand{\opponentOneJobPost}       % Оппонент 1, должность
{\todo{старший научный сотрудник}}

\newcommand{\opponentTwoFio}           % Оппонент 2, ФИО
{\todo{Фамилия Имя Отчество}}
\newcommand{\opponentTwoRegalia}       % Оппонент 2, регалии
{\todo{кандидат физико-математических наук}}
\newcommand{\opponentTwoJobPlace}      % Оппонент 2, место работы
{\todo{Основное место работы c длинным длинным длинным длинным названием}}
\newcommand{\opponentTwoJobPost}       % Оппонент 2, должность
{\todo{старший научный сотрудник}}

\newcommand{\leadingOrganizationTitle} % Ведущая организация, дополнительные строки
{\todo{Федеральное государственное бюджетное образовательное учреждение высшего профессионального образования с~длинным длинным длинным длинным названием}}

\newcommand{\defenseDate}              % Защита, дата
{\todo{DD mmmmmmmm YYYY~г.~в~XX часов}}
\newcommand{\defenseCouncilNumber}     % Защита, номер диссертационного совета
{\todo{NN}}
\newcommand{\defenseCouncilTitle}      % Защита, учреждение диссертационного совета
{\todo{Название учреждения}}
\newcommand{\defenseCouncilAddress}    % Защита, адрес учреждение диссертационного совета
{\todo{Адрес}}

\newcommand{\defenseSecretaryFio}      % Секретарь диссертационного совета, ФИО
{\todo{Фамилия Имя Отчество}}
\newcommand{\defenseSecretaryRegalia}  % Секретарь диссертационного совета, регалии
{\todo{д-р~физ.-мат. наук}}            % Для сокращений есть ГОСТы, например: ГОСТ Р 7.0.12-2011 + http://base.garant.ru/179724/#block_30000

\newcommand{\synopsisLibrary}          % Автореферат, название библиотеки
{\todo{Название библиотеки}}
\newcommand{\synopsisDate}             % Автореферат, дата рассылки
{\todo{DD mmmmmmmm YYYY года}}

\newcommand{\keywords}%                 % Ключевые слова для метаданных PDF диссертации и автореферата
{}      % Основные сведения
%%%%%%%%%%%%%%%%%%%%%%%%%%%%%%%%%%%%%%%%%%%%%%%%%%%%%%%%%%%%%%%%%%%%%%%%%%%
%%%%%%%%%%%%%%%%%%%%%%%%%%%%%%%%%%%%%%%%%%%%%%%%%%%%%%%%%%%%%%%%%%%%%%%%%%%
%%%% Макет страницы %%%
% Выставляем значения полей (ГОСТ 7.0.11-2011, 5.3.7)
\geometry{a4paper,top=2cm,bottom=2cm,left=2.5cm,right=1cm}

%%% Кодировки и шрифты %%%
\ifxetexorluatex
    \setmainlanguage[babelshorthands=true]{russian}  % Язык по-умолчанию русский с поддержкой приятных команд пакета babel
    \setotherlanguage{english}                       % Дополнительный язык = английский (в американской вариации по-умолчанию)
    \ifXeTeX
        \defaultfontfeatures{Ligatures=TeX,Mapping=tex-text}
    \else
        \defaultfontfeatures{Ligatures=TeX}
    \fi
    \setmainfont{Times New Roman}
    \newfontfamily\cyrillicfont{Times New Roman}
    \setsansfont{Arial}
    \newfontfamily\cyrillicfontsf{Arial}
    \setmonofont{Courier New}
    \newfontfamily\cyrillicfonttt{Courier New}
\else
    \IfFileExists{pscyr.sty}{\renewcommand{\rmdefault}{ftm}}{}
\fi

%%% Интервалы %%%
%linespread-реализация ближе к реализации полуторного интервала в ворде.
%setspace реализация заточена под шрифты 10, 11, 12pt, под остальные кегли хуже, но всё же ближе к типографской классике. 
%\linespread{1.3}                    % Полуторный интервал (ГОСТ Р 7.0.11-2011, 5.3.6)

%%% Выравнивание и переносы %%%
\sloppy                             % Избавляемся от переполнений
\clubpenalty=10000                  % Запрещаем разрыв страницы после первой строки абзаца
\widowpenalty=10000                 % Запрещаем разрыв страницы после последней строки абзаца

%%% Подписи %%%
\captionsetup{%
singlelinecheck=off,                % Многострочные подписи, например у таблиц
skip=2pt,                           % Вертикальная отбивка между подписью и содержимым рисунка или таблицы определяется ключом
justification=centering,            % Центрирование подписей, заданных командой \caption
}

%%% Рисунки %%%
\DeclareCaptionLabelSeparator*{emdash}{~--- }             % (ГОСТ 2.105, 4.3.1)
\captionsetup[figure]{labelsep=emdash,font=onehalfspacing,position=bottom}

%%% Таблицы %%%
\ifthenelse{\equal{\thetabcap}{0}}{%
    \newcommand{\tabcapalign}{\raggedright}  % по левому краю страницы или аналога parbox
}

\ifthenelse{\equal{\thetablaba}{0} \AND \equal{\thetabcap}{1}}{%
    \newcommand{\tabcapalign}{\raggedright}  % по левому краю страницы или аналога parbox
}

\ifthenelse{\equal{\thetablaba}{1} \AND \equal{\thetabcap}{1}}{%
    \newcommand{\tabcapalign}{\centering}    % по центру страницы или аналога parbox
}

\ifthenelse{\equal{\thetablaba}{2} \AND \equal{\thetabcap}{1}}{%
    \newcommand{\tabcapalign}{\raggedleft}   % по правому краю страницы или аналога parbox
}

\ifthenelse{\equal{\thetabtita}{0} \AND \equal{\thetabcap}{1}}{%
    \newcommand{\tabtitalign}{\raggedright}  % по левому краю страницы или аналога parbox
}

\ifthenelse{\equal{\thetabtita}{1} \AND \equal{\thetabcap}{1}}{%
    \newcommand{\tabtitalign}{\centering}    % по центру страницы или аналога parbox
}

\ifthenelse{\equal{\thetabtita}{2} \AND \equal{\thetabcap}{1}}{%
    \newcommand{\tabtitalign}{\raggedleft}   % по правому краю страницы или аналога parbox
}

\DeclareCaptionFormat{tablenocaption}{\tabcapalign #1\strut}        % Наименование таблицы отсутствует
\ifthenelse{\equal{\thetabcap}{0}}{%
    \DeclareCaptionFormat{tablecaption}{\tabcapalign #1#2#3}
    \captionsetup[table]{labelsep=emdash}                       % тире как разделитель идентификатора с номером от наименования
}{%
    \DeclareCaptionFormat{tablecaption}{\tabcapalign #1#2\par%  % Идентификатор таблицы на отдельной строке
        \tabtitalign{#3}}                                       % Наименование таблицы строкой ниже
    \captionsetup[table]{labelsep=space}                        % пробельный разделитель идентификатора с номером от наименования
}
\captionsetup[table]{format=tablecaption,singlelinecheck=off,font=onehalfspacing,position=top,skip=0pt}  % многострочные наименования и прочее
\DeclareCaptionLabelFormat{continued}{Продолжение таблицы~#2}

%%% Подписи подрисунков %%%
\renewcommand{\thesubfigure}{\asbuk{subfigure}}           % Буквенные номера подрисунков
\captionsetup[subfigure]{font={normalsize},               % Шрифт подписи названий подрисунков (не отличается от основного)
    labelformat=brace,                                    % Формат обозначения подрисунка
    justification=centering,                              % Выключка подписей (форматирование), один из вариантов            
}
%\DeclareCaptionFont{font12pt}{\fontsize{12pt}{13pt}\selectfont} % объявляем шрифт 12pt для использования в подписях, тут же надо интерлиньяж объявлять, если не наследуется
%\captionsetup[subfigure]{font={font12pt}}                 % Шрифт подписи названий подрисунков (всегда 12pt)

%%% Настройки гиперссылок %%%
\ifLuaTeX
    \hypersetup{
        unicode,                % Unicode encoded PDF strings
    }
\fi

\hypersetup{
    linktocpage=true,           % ссылки с номера страницы в оглавлении, списке таблиц и списке рисунков
%    linktoc=all,                % both the section and page part are links
%    pdfpagelabels=false,        % set PDF page labels (true|false)
    plainpages=false,           % Forces page anchors to be named by the Arabic form  of the page number, rather than the formatted form
    colorlinks,                 % ссылки отображаются раскрашенным текстом, а не раскрашенным прямоугольником, вокруг текста
    linkcolor={linkcolor},      % цвет ссылок типа ref, eqref и подобных
    citecolor={citecolor},      % цвет ссылок-цитат
    urlcolor={urlcolor},        % цвет гиперссылок
%    hidelinks,                  % Hide links (removing color and border)
    pdftitle={\thesisTitle},    % Заголовок
    pdfauthor={\thesisAuthor},  % Автор
    pdfsubject={\thesisSpecialtyNumber\ \thesisSpecialtyTitle},      % Тема
%    pdfcreator={Создатель},     % Создатель, Приложение
%    pdfproducer={Производитель},% Производитель, Производитель PDF
    pdfkeywords={\keywords},    % Ключевые слова
    pdflang={ru},
}

%%% Шаблон %%%
\DeclareRobustCommand{\todo}{\textcolor{red}}       % решаем проблему превращения названия цвета в результате \MakeUppercase, http://tex.stackexchange.com/a/187930/79756 , \DeclareRobustCommand protects \todo from expanding inside \MakeUppercase
\setlength{\parindent}{2.5em}                       % Абзацный отступ. Должен быть одинаковым по всему тексту и равен пяти знакам (ГОСТ Р 7.0.11-2011, 5.3.7).

%%% Списки %%%
% Используем дефис для ненумерованных списков (ГОСТ 2.105-95, 4.1.7)
\renewcommand{\labelitemi}{\normalfont\bfseries{--}} 
\setlist{nosep,%                                    % Единый стиль для всех списков (пакет enumitem), без дополнительных интервалов.
    labelindent=\parindent,leftmargin=*%            % Каждый пункт, подпункт и перечисление записывают с абзацного отступа (ГОСТ 2.105-95, 4.1.8)
}
    % Стили общие для диссертации и автореферата
%%% Макет страницы %%%
% Выставляем значения полей (ГОСТ 7.0.11-2011, 5.3.7)
\geometry{a4paper,top=2cm,bottom=2cm,left=2.5cm,right=1cm}

%%% Кодировки и шрифты %%%
\ifxetexorluatex
    \setmainlanguage[babelshorthands=true]{russian}  % Язык по-умолчанию русский с поддержкой приятных команд пакета babel
    \setotherlanguage{english}                       % Дополнительный язык = английский (в американской вариации по-умолчанию)
    \ifXeTeX
        \defaultfontfeatures{Ligatures=TeX,Mapping=tex-text}
    \else
        \defaultfontfeatures{Ligatures=TeX}
    \fi
    \setmainfont{Times New Roman}
    \newfontfamily\cyrillicfont{Times New Roman}
    \setsansfont{Arial}
    \newfontfamily\cyrillicfontsf{Arial}
    \setmonofont{Courier New}
    \newfontfamily\cyrillicfonttt{Courier New}
\else
    \IfFileExists{pscyr.sty}{\renewcommand{\rmdefault}{ftm}}{}
\fi

%%% Интервалы %%%
%linespread-реализация ближе к реализации полуторного интервала в ворде.
%setspace реализация заточена под шрифты 10, 11, 12pt, под остальные кегли хуже, но всё же ближе к типографской классике. 
%\linespread{1.3}                    % Полуторный интервал (ГОСТ Р 7.0.11-2011, 5.3.6)

%%% Выравнивание и переносы %%%
\sloppy                             % Избавляемся от переполнений
\clubpenalty=10000                  % Запрещаем разрыв страницы после первой строки абзаца
\widowpenalty=10000                 % Запрещаем разрыв страницы после последней строки абзаца

%%% Изображения %%%
\graphicspath{{../images/}{images/}}         % Пути к изображениям

%%% Подписи %%%
\captionsetup{%
singlelinecheck=off,                % Многострочные подписи, например у таблиц
skip=2pt,                           % Вертикальная отбивка между подписью и содержимым рисунка или таблицы определяется ключом
justification=centering,            % Центрирование подписей, заданных командой \caption
}

%%% Рисунки %%%
\DeclareCaptionLabelSeparator*{emdash}{~--- }             % (ГОСТ 2.105, 4.3.1)
\captionsetup[figure]{labelsep=emdash,font=onehalfspacing,position=bottom}

%%% Таблицы %%%
\ifthenelse{\equal{\thetabcap}{0}}{%
    \newcommand{\tabcapalign}{\raggedright}  % по левому краю страницы или аналога parbox
}

\ifthenelse{\equal{\thetablaba}{0} \AND \equal{\thetabcap}{1}}{%
    \newcommand{\tabcapalign}{\raggedright}  % по левому краю страницы или аналога parbox
}

\ifthenelse{\equal{\thetablaba}{1} \AND \equal{\thetabcap}{1}}{%
    \newcommand{\tabcapalign}{\centering}    % по центру страницы или аналога parbox
}

\ifthenelse{\equal{\thetablaba}{2} \AND \equal{\thetabcap}{1}}{%
    \newcommand{\tabcapalign}{\raggedleft}   % по правому краю страницы или аналога parbox
}

\ifthenelse{\equal{\thetabtita}{0} \AND \equal{\thetabcap}{1}}{%
    \newcommand{\tabtitalign}{\raggedright}  % по левому краю страницы или аналога parbox
}

\ifthenelse{\equal{\thetabtita}{1} \AND \equal{\thetabcap}{1}}{%
    \newcommand{\tabtitalign}{\centering}    % по центру страницы или аналога parbox
}

\ifthenelse{\equal{\thetabtita}{2} \AND \equal{\thetabcap}{1}}{%
    \newcommand{\tabtitalign}{\raggedleft}   % по правому краю страницы или аналога parbox
}

\DeclareCaptionFormat{tablenocaption}{\tabcapalign #1\strut}        % Наименование таблицы отсутствует
\ifthenelse{\equal{\thetabcap}{0}}{%
    \DeclareCaptionFormat{tablecaption}{\tabcapalign #1#2#3}
    \captionsetup[table]{labelsep=emdash}                       % тире как разделитель идентификатора с номером от наименования
}{%
    \DeclareCaptionFormat{tablecaption}{\tabcapalign #1#2\par%  % Идентификатор таблицы на отдельной строке
        \tabtitalign{#3}}                                       % Наименование таблицы строкой ниже
    \captionsetup[table]{labelsep=space}                        % пробельный разделитель идентификатора с номером от наименования
}
\captionsetup[table]{format=tablecaption,singlelinecheck=off,font=onehalfspacing,position=top,skip=0pt}  % многострочные наименования и прочее
\DeclareCaptionLabelFormat{continued}{Продолжение таблицы~#2}

%%% Подписи подрисунков %%%
\renewcommand{\thesubfigure}{\asbuk{subfigure}}           % Буквенные номера подрисунков
\captionsetup[subfigure]{font={normalsize},               % Шрифт подписи названий подрисунков (не отличается от основного)
    labelformat=brace,                                    % Формат обозначения подрисунка
    justification=centering,                              % Выключка подписей (форматирование), один из вариантов            
}
%\DeclareCaptionFont{font12pt}{\fontsize{12pt}{13pt}\selectfont} % объявляем шрифт 12pt для использования в подписях, тут же надо интерлиньяж объявлять, если не наследуется
%\captionsetup[subfigure]{font={font12pt}}                 % Шрифт подписи названий подрисунков (всегда 12pt)

%%% Настройки гиперссылок %%%
\ifLuaTeX
    \hypersetup{
        unicode,                % Unicode encoded PDF strings
    }
\fi

\hypersetup{
    linktocpage=true,           % ссылки с номера страницы в оглавлении, списке таблиц и списке рисунков
%    linktoc=all,                % both the section and page part are links
%    pdfpagelabels=false,        % set PDF page labels (true|false)
    plainpages=false,           % Forces page anchors to be named by the Arabic form  of the page number, rather than the formatted form
    colorlinks,                 % ссылки отображаются раскрашенным текстом, а не раскрашенным прямоугольником, вокруг текста
    linkcolor={linkcolor},      % цвет ссылок типа ref, eqref и подобных
    citecolor={citecolor},      % цвет ссылок-цитат
    urlcolor={urlcolor},        % цвет гиперссылок
%    hidelinks,                  % Hide links (removing color and border)
    pdftitle={\thesisTitle},    % Заголовок
    pdfauthor={\thesisAuthor},  % Автор
    pdfsubject={\thesisSpecialtyNumber\ \thesisSpecialtyTitle},      % Тема
%    pdfcreator={Создатель},     % Создатель, Приложение
%    pdfproducer={Производитель},% Производитель, Производитель PDF
    pdfkeywords={\keywords},    % Ключевые слова
    pdflang={ru},
    draft,
}

%%% Шаблон %%%
\DeclareRobustCommand{\todo}{\textcolor{red}}       % решаем проблему превращения названия цвета в результате \MakeUppercase, http://tex.stackexchange.com/a/187930/79756 , \DeclareRobustCommand protects \todo from expanding inside \MakeUppercase
\setlength{\parindent}{2.5em}                       % Абзацный отступ. Должен быть одинаковым по всему тексту и равен пяти знакам (ГОСТ Р 7.0.11-2011, 5.3.7).

%%% Списки %%%
% Используем дефис для ненумерованных списков (ГОСТ 2.105-95, 4.1.7)
\renewcommand{\labelitemi}{\normalfont\bfseries{--}} 
\setlist{nosep,%                                    % Единый стиль для всех списков (пакет enumitem), без дополнительных интервалов.
    labelindent=\parindent,leftmargin=*%            % Каждый пункт, подпункт и перечисление записывают с абзацного отступа (ГОСТ 2.105-95, 4.1.8)
}
%%%%%%%%%%%%%%%%%%%%%%%%%%%%%%%%%%%%%%%%%%%%%%%%%%%%%%%%%%%%%%%%%%%%%%%%%%%
%%%%%%%%%%%%%%%%%%%%%%%%%%%%%%%%%%%%%%%%%%%%%%%%%%%%%%%%%%%%%%%%%%%%%%%%%%%
%%%% Изображения %%%
\graphicspath{{images/}{Dissertation/images/}}         % Пути к изображениям

\LoadInterface {titlesec}                   % Подгружаем интерфейсы для дополнительных опций управления некоторыми пакетами

%%% Блок управления параметрами для выравнивания заголовков в тексте %%%
\newlength{\otstuplen}
\setlength{\otstuplen}{\theotstup\parindent}
\ifthenelse{\equal{\theheadingalign}{0}}{% выравнивание заголовков в тексте
    \newcommand{\hdngalign}{\filcenter}                % по центру
    \newcommand{\hdngaligni}{\hfill\hspace{\otstuplen}}% по центру
}{%
    \newcommand{\hdngalign}{\filright}                 % по левому краю
    \newcommand{\hdngaligni}{\hspace{\otstuplen}}      % по левому краю
} % В обоих случаях вроде бы без переноса, как и надо (ГОСТ Р 7.0.11-2011, 5.3.5)

%%% Оглавление %%%
\renewcommand{\cftchapdotsep}{\cftdotsep}                % отбивка точками до номера страницы начала главы/раздела
\renewcommand{\cfttoctitlefont}{\hdngaligni\fontsize{14pt}{16pt}\selectfont\bfseries}% вместе со следующей строкой
\renewcommand{\cftaftertoctitle}{\hfill}                 % устанавливает заголовок по центру
\setlength{\cftbeforetoctitleskip}{-1.4\curtextsize}     % Поскольку этот заголовок всегда является первым на странице, то перед ним отделять пустым тройным интервалом не следует. Независимо от основного шрифта, в этом случае зануление (почти) происходит при -1.4\curtextsize.
\setlength{\cftaftertoctitleskip}{\theintvl\curtextsize} % Если считаем Оглавление заголовком, то выставляем после него тройной интервал через наше определённое значение

%% Переносить слова в заголовке не допускается (ГОСТ Р 7.0.11-2011, 5.3.5). Заголовки в оглавлении должны точно повторять заголовки в тексте (ГОСТ Р 7.0.11-2011, 5.2.3). Прямого указания на запрет переносов в оглавлении нет, но по той же логике невнесения искажений в смысл, лучше в оглавлении не переносить:
\cftsetrmarg{2.55em plus1fil}                       %To have the (sectional) titles in the ToC, etc., typeset ragged right with no hyphenation
\renewcommand{\cftchappagefont}{\normalfont}        % нежирные номера страниц у глав в оглавлении
\renewcommand{\cftchapleader}{\cftdotfill{\cftchapdotsep}}% нежирные точки до номеров страниц у глав в оглавлении
%\renewcommand{\cftchapfont}{}                       % нежирные названия глав в оглавлении

\ifthenelse{\theheadingdelim > 0}{%
    \renewcommand\cftchapaftersnum{.\ }   % добавляет точку с пробелом после номера раздела в оглавлении
}{%
\renewcommand\cftchapaftersnum{\quad}     % добавляет \quad после номера раздела в оглавлении
}
\ifthenelse{\theheadingdelim > 1}{%
    \renewcommand\cftsecaftersnum{.\ }    % добавляет точку с пробелом после номера подраздела в оглавлении
    \renewcommand\cftsubsecaftersnum{.\ } % добавляет точку с пробелом после номера подподраздела в оглавлении
}{%
\renewcommand\cftsecaftersnum{\quad}      % добавляет \quad после номера подраздела в оглавлении
\renewcommand\cftsubsecaftersnum{\quad}   % добавляет \quad после номера подподраздела в оглавлении
}

\ifthenelse{\equal{\thepgnum}{1}}{%
    \addtocontents{toc}{~\hfill{Стр.}\par}% добавить Стр. над номерами страниц
}

%%% Оформление названий глав %%%
%% настройки заголовка списка рисунков
\renewcommand{\cftloftitlefont}{\hdngaligni\fontsize{14pt}{16pt}\selectfont\bfseries}% вместе со следующей строкой
\renewcommand{\cftafterloftitle}{\hfill}                                             % устанавливает заголовок по центру
\setlength{\cftbeforeloftitleskip}{-1.5\curtextsize}     % Поскольку этот заголовок всегда является первым на странице, то перед ним отделять пустым тройным интервалом не следует. Независимо от основного шрифта, в этом случае зануление (почти) происходит при -1.5\curtextsize.
\setlength{\cftafterloftitleskip}{\theintvl\curtextsize} % выставляем после него тройной интервал через наше определённое значение

%% настройки заголовка списка таблиц
\renewcommand{\cftlottitlefont}{\hdngaligni\fontsize{14pt}{16pt}\selectfont\bfseries}% вместе со следующей строкой
\renewcommand{\cftafterlottitle}{\hfill}                                             % устанавливает заголовок по центру
\setlength{\cftbeforelottitleskip}{-1.5\curtextsize}     % Поскольку этот заголовок всегда является первым на странице, то перед ним отделять пустым тройным интервалом не следует. Независимо от основного шрифта, в этом случае зануление (почти) происходит при -1.5\curtextsize.
\setlength{\cftafterlottitleskip}{\theintvl\curtextsize} % выставляем после него тройной интервал через наше определённое значение

\ifnum\curtextsize>\bigtextsize     % Проверяем условие использования базового шрифта 14 pt
\setlength{\headheight}{17pt}       % Исправляем высоту заголовка
\else
\setlength{\headheight}{15pt}       % Исправляем высоту заголовка
\fi

%%% Колонтитулы %%%
% Порядковый номер страницы печатают на середине верхнего поля страницы (ГОСТ Р 7.0.11-2011, 5.3.8)
\makeatletter
\let\ps@plain\ps@fancy              % Подчиняем первые страницы каждой главы общим правилам
\makeatother
\pagestyle{fancy}                   % Меняем стиль оформления страниц
\fancyhf{}                          % Очищаем текущие значения
\fancyhead[C]{\thepage}             % Печатаем номер страницы на середине верхнего поля
\renewcommand{\headrulewidth}{0pt}  % Убираем разделительную линию

%%% Оформление заголовков глав, разделов, подразделов %%%
%% Работа должна быть выполнена ... размером шрифта 12-14 пунктов (ГОСТ Р 7.0.11-2011, 5.3.8). То есть не должно быть надписей шрифтом более 14. Так и поставим.
%% Эти установки будут давать одинаковый результат независимо от выбора базовым шрифтом 12 пт или 14 пт
\titleformat{\chapter}[block]                                % default display;  hang = with a hanging label. (Like the standard \section.); block = typesets the whole title in a block (a paragraph) without additional formatting. Useful in centered titles
        {\hdngalign\fontsize{14pt}{16pt}\selectfont\bfseries}% 
        %\fontsize{<size>}{<skip>} % второе число ставим 1.2*первое, чтобы адекватно отрабатывали команды по расчету полуторного интервала (домножая разные комбинации коэффициентов на этот)
        {\thechapter\cftchapaftersnum}                       % Заголовки в оглавлении должны точно повторять заголовки в тексте (ГОСТ Р 7.0.11-2011, 5.2.3).
        {0em}% отступ от номера до текста
        {}%

\titleformat{\section}[block]                                % default hang;  hang = with a hanging label. (Like the standard \section.); block = typesets the whole title in a block (a paragraph) without additional formatting. Useful in centered titles
        {\hdngalign\fontsize{14pt}{16pt}\selectfont\bfseries}% 
        %\fontsize{<size>}{<skip>} % второе число ставим 1.2*первое, чтобы адекватно отрабатывали команды по расчету полуторного интервала (домножая разные комбинации коэффициентов на этот)
        {\thesection\cftsecaftersnum}                        % Заголовки в оглавлении должны точно повторять заголовки в тексте (ГОСТ Р 7.0.11-2011, 5.2.3).
        {0em}% отступ от номера до текста
        {}%

\titleformat{\subsection}[block]                             % default hang;  hang = with a hanging label. (Like the standard \section.); block = typesets the whole title in a block (a paragraph) without additional formatting. Useful in centered titles
        {\hdngalign\fontsize{14pt}{16pt}\selectfont\bfseries}% 
        %\fontsize{<size>}{<skip>} % второе число ставим 1.2*первое, чтобы адекватно отрабатывали команды по расчету полуторного интервала (домножая разные комбинации коэффициентов на этот)
        {\thesubsection\cftsubsecaftersnum}                  % Заголовки в оглавлении должны точно повторять заголовки в тексте (ГОСТ Р 7.0.11-2011, 5.2.3).
        {0em}% отступ от номера до текста
        {}%

\ifthenelse{\equal{\thechapstyle}{1}}{%
    \sectionformat{\chapter}{% Параметры заголовков разделов в тексте
        label=\chaptername\ \thechapter\cftchapaftersnum,
        labelsep=0em,
    }
    %% Следующие две строки: будет вписано слово Глава перед каждым номером раздела в оглавлении   
    \renewcommand{\cftchappresnum}{\chaptername\ }
    \setlength{\cftchapnumwidth}{\widthof{\cftchapfont\cftchappresnum\thechapter\cftchapaftersnum}}
}%

%% Интервалы между заголовками
% На эти величины titlespacing множит через *
\beforetitleunit=\curtextsize% привязались к нашему размеру шрифта
\aftertitleunit=\curtextsize% привязались к нашему размеру шрифта

% Счётчик intvl и длина \otstup определены в файле setup
\titlespacing{\chapter}{\theotstup\parindent}{-1.7em}{*\theintvl}       % Заголовки отделяют от текста сверху и снизу тремя интервалами (ГОСТ Р 7.0.11-2011, 5.3.5). Поскольку название главы всегда является первым на странице, то перед ним отделять пустым тройным интервалом не следует. Независимо от основного шрифта, в этом случае зануление происходит при -1.7em.
\titlespacing{\section}{\theotstup\parindent}{*\theintvl}{*\theintvl}
\titlespacing{\subsection}{\theotstup\parindent}{*\theintvl}{*\theintvl}
\titlespacing{\subsubsection}{\theotstup\parindent}{*\theintvl}{*\theintvl}

%%% Блок дополнительного управления размерами заголовков
\ifthenelse{\equal{\theheadingsize}{1}}{% Пропорциональные заголовки и базовый шрифт 14 пт
    \renewcommand{\cfttoctitlefont}{\hdngaligni\Large\bfseries} % Исправляем размер заголовка оглавления
    \setlength{\cftbeforetoctitleskip}{-1.2\curtextsize}        % Исправляем вертикальный отступ перед заголовком оглавления
    \renewcommand{\cftloftitlefont}{\hdngaligni\Large\bfseries} % Исправляем размер заголовка списка рисунков
    \setlength{\cftbeforeloftitleskip}{-1.4\curtextsize}        % Исправляем вертикальный отступ перед заголовком списка рисунков
    \renewcommand{\cftlottitlefont}{\hdngaligni\Large\bfseries} % Исправляем размер заголовка списка таблиц 
    \setlength{\cftbeforelottitleskip}{-1.4\curtextsize}        % Исправляем вертикальный отступ перед заголовком списка таблиц
    \sectionformat{\chapter}{% Параметры заголовков разделов в тексте
        format=\hdngalign\Large\bfseries, % Исправляем размер заголовка
        top-=0.4em,                       % Исправляем вертикальный отступ перед заголовком
    }
    \sectionformat{\section}{% Параметры заголовков подразделов в тексте
        format=\hdngalign\large\bfseries, % Исправляем размер заголовка
    }
}

\ifthenelse{\equal{\theheadingsize}{1}\AND \curtextsize < \bigtextsize}{% Пропорциональные заголовки и базовый шрифт 14 пт
    \sectionformat{\chapter}{% Параметры заголовков разделов в тексте
        top-=0.2em, % Исправляем вертикальный отступ перед заголовком
    }
}

%%% Счётчики %%%

%% Упрощённые настройки шаблона диссертации: нумерация формул, таблиц, рисунков
\ifthenelse{\equal{\thecontnumeq}{1}}{%
    \counterwithout{equation}{chapter} % Убираем связанность номера формулы с номером главы/раздела
}
\ifthenelse{\equal{\thecontnumfig}{1}}{%
    \counterwithout{figure}{chapter}   % Убираем связанность номера рисунка с номером главы/раздела
}
\ifthenelse{\equal{\thecontnumtab}{1}}{%
    \counterwithout{table}{chapter}    % Убираем связанность номера таблицы с номером главы/раздела
}


%%http://www.linux.org.ru/forum/general/6993203#comment-6994589 (используется totcount)
\makeatletter
\def\formbytotal#1#2#3#4#5{%
    \newcount\@c
    \@c\totvalue{#1}\relax
    \newcount\@last
    \newcount\@pnul
    \@last\@c\relax
    \divide\@last 10
    \@pnul\@last\relax
    \divide\@pnul 10
    \multiply\@pnul-10
    \advance\@pnul\@last
    \multiply\@last-10
    \advance\@last\@c
    \total{#1}~#2%
    \ifnum\@pnul=1#5\else%
    \ifcase\@last#5\or#3\or#4\or#4\or#4\else#5\fi
    \fi
}
\makeatother

\AtBeginDocument{
%% регистрируем счётчики в системе totcounter
    \regtotcounter{totalcount@figure}
    \regtotcounter{totalcount@table}       % Если иным способом поставить в преамбуле то ошибка в числе таблиц
    \regtotcounter{TotPages}               % Если иным способом поставить в преамбуле то ошибка в числе страниц
}           % Стили для диссертации
\LoadInterface {titlesec}                   % Подгружаем интерфейсы для дополнительных опций управления некоторыми пакетами

%%% Блок управления параметрами для выравнивания заголовков в тексте %%%
\newlength{\otstuplen}
\setlength{\otstuplen}{\theotstup\parindent}
\ifthenelse{\equal{\theheadingalign}{0}}{% выравнивание заголовков в тексте
    \newcommand{\hdngalign}{\filcenter}                % по центру
    \newcommand{\hdngaligni}{\hfill\hspace{\otstuplen}}% по центру
}{%
    \newcommand{\hdngalign}{\filright}                 % по левому краю
    \newcommand{\hdngaligni}{\hspace{\otstuplen}}      % по левому краю
} % В обоих случаях вроде бы без переноса, как и надо (ГОСТ Р 7.0.11-2011, 5.3.5)

%%% Оглавление %%%
\renewcommand{\cftchapdotsep}{\cftdotsep}                % отбивка точками до номера страницы начала главы/раздела
\renewcommand{\cfttoctitlefont}{\hdngaligni\fontsize{14pt}{16pt}\selectfont\bfseries}% вместе со следующей строкой
\renewcommand{\cftaftertoctitle}{\hfill}                 % устанавливает заголовок по центру
\setlength{\cftbeforetoctitleskip}{-1.4\curtextsize}     % Поскольку этот заголовок всегда является первым на странице, то перед ним отделять пустым тройным интервалом не следует. Независимо от основного шрифта, в этом случае зануление (почти) происходит при -1.4\curtextsize.
\setlength{\cftaftertoctitleskip}{\theintvl\curtextsize} % Если считаем Оглавление заголовком, то выставляем после него тройной интервал через наше определённое значение

%% Переносить слова в заголовке не допускается (ГОСТ Р 7.0.11-2011, 5.3.5). Заголовки в оглавлении должны точно повторять заголовки в тексте (ГОСТ Р 7.0.11-2011, 5.2.3). Прямого указания на запрет переносов в оглавлении нет, но по той же логике невнесения искажений в смысл, лучше в оглавлении не переносить:
\cftsetrmarg{2.55em plus1fil}                       %To have the (sectional) titles in the ToC, etc., typeset ragged right with no hyphenation
\renewcommand{\cftchappagefont}{\normalfont}        % нежирные номера страниц у глав в оглавлении
\renewcommand{\cftchapleader}{\cftdotfill{\cftchapdotsep}}% нежирные точки до номеров страниц у глав в оглавлении
%\renewcommand{\cftchapfont}{}                       % нежирные названия глав в оглавлении

\ifthenelse{\theheadingdelim > 0}{%
    \renewcommand\cftchapaftersnum{.\ }   % добавляет точку с пробелом после номера раздела в оглавлении
}{%
\renewcommand\cftchapaftersnum{\quad}     % добавляет \quad после номера раздела в оглавлении
}
\ifthenelse{\theheadingdelim > 1}{%
    \renewcommand\cftsecaftersnum{.\ }    % добавляет точку с пробелом после номера подраздела в оглавлении
    \renewcommand\cftsubsecaftersnum{.\ } % добавляет точку с пробелом после номера подподраздела в оглавлении
}{%
\renewcommand\cftsecaftersnum{\quad}      % добавляет \quad после номера подраздела в оглавлении
\renewcommand\cftsubsecaftersnum{\quad}   % добавляет \quad после номера подподраздела в оглавлении
}

\ifthenelse{\equal{\thepgnum}{1}}{%
    \addtocontents{toc}{~\hfill{Стр.}\par}% добавить Стр. над номерами страниц
}

%%% Оформление названий глав %%%
%% настройки заголовка списка рисунков
\renewcommand{\cftloftitlefont}{\hdngaligni\fontsize{14pt}{16pt}\selectfont\bfseries}% вместе со следующей строкой
\renewcommand{\cftafterloftitle}{\hfill}                                             % устанавливает заголовок по центру
\setlength{\cftbeforeloftitleskip}{-1.5\curtextsize}     % Поскольку этот заголовок всегда является первым на странице, то перед ним отделять пустым тройным интервалом не следует. Независимо от основного шрифта, в этом случае зануление (почти) происходит при -1.5\curtextsize.
\setlength{\cftafterloftitleskip}{\theintvl\curtextsize} % выставляем после него тройной интервал через наше определённое значение

%% настройки заголовка списка таблиц
\renewcommand{\cftlottitlefont}{\hdngaligni\fontsize{14pt}{16pt}\selectfont\bfseries}% вместе со следующей строкой
\renewcommand{\cftafterlottitle}{\hfill}                                             % устанавливает заголовок по центру
\setlength{\cftbeforelottitleskip}{-1.5\curtextsize}     % Поскольку этот заголовок всегда является первым на странице, то перед ним отделять пустым тройным интервалом не следует. Независимо от основного шрифта, в этом случае зануление (почти) происходит при -1.5\curtextsize.
\setlength{\cftafterlottitleskip}{\theintvl\curtextsize} % выставляем после него тройной интервал через наше определённое значение

\ifnum\curtextsize>\bigtextsize     % Проверяем условие использования базового шрифта 14 pt
\setlength{\headheight}{17pt}       % Исправляем высоту заголовка
\else
\setlength{\headheight}{15pt}       % Исправляем высоту заголовка
\fi

%%% Колонтитулы %%%
% Порядковый номер страницы печатают на середине верхнего поля страницы (ГОСТ Р 7.0.11-2011, 5.3.8)
\makeatletter
\let\ps@plain\ps@fancy              % Подчиняем первые страницы каждой главы общим правилам
\makeatother
\pagestyle{fancy}                   % Меняем стиль оформления страниц
\fancyhf{}                          % Очищаем текущие значения
\fancyhead[C]{\thepage}             % Печатаем номер страницы на середине верхнего поля
\renewcommand{\headrulewidth}{0pt}  % Убираем разделительную линию

%%% Оформление заголовков глав, разделов, подразделов %%%
%% Работа должна быть выполнена ... размером шрифта 12-14 пунктов (ГОСТ Р 7.0.11-2011, 5.3.8). То есть не должно быть надписей шрифтом более 14. Так и поставим.
%% Эти установки будут давать одинаковый результат независимо от выбора базовым шрифтом 12 пт или 14 пт
\titleformat{\chapter}[block]                                % default display;  hang = with a hanging label. (Like the standard \section.); block = typesets the whole title in a block (a paragraph) without additional formatting. Useful in centered titles
        {\hdngalign\fontsize{14pt}{16pt}\selectfont\bfseries}% 
        %\fontsize{<size>}{<skip>} % второе число ставим 1.2*первое, чтобы адекватно отрабатывали команды по расчету полуторного интервала (домножая разные комбинации коэффициентов на этот)
        {\thechapter\cftchapaftersnum}                       % Заголовки в оглавлении должны точно повторять заголовки в тексте (ГОСТ Р 7.0.11-2011, 5.2.3).
        {0em}% отступ от номера до текста
        {}%

\titleformat{\section}[block]                                % default hang;  hang = with a hanging label. (Like the standard \section.); block = typesets the whole title in a block (a paragraph) without additional formatting. Useful in centered titles
        {\hdngalign\fontsize{14pt}{16pt}\selectfont\bfseries}% 
        %\fontsize{<size>}{<skip>} % второе число ставим 1.2*первое, чтобы адекватно отрабатывали команды по расчету полуторного интервала (домножая разные комбинации коэффициентов на этот)
        {\thesection\cftsecaftersnum}                        % Заголовки в оглавлении должны точно повторять заголовки в тексте (ГОСТ Р 7.0.11-2011, 5.2.3).
        {0em}% отступ от номера до текста
        {}%

\titleformat{\subsection}[block]                             % default hang;  hang = with a hanging label. (Like the standard \section.); block = typesets the whole title in a block (a paragraph) without additional formatting. Useful in centered titles
        {\hdngalign\fontsize{14pt}{16pt}\selectfont\bfseries}% 
        %\fontsize{<size>}{<skip>} % второе число ставим 1.2*первое, чтобы адекватно отрабатывали команды по расчету полуторного интервала (домножая разные комбинации коэффициентов на этот)
        {\thesubsection\cftsubsecaftersnum}                  % Заголовки в оглавлении должны точно повторять заголовки в тексте (ГОСТ Р 7.0.11-2011, 5.2.3).
        {0em}% отступ от номера до текста
        {}%

\ifthenelse{\equal{\thechapstyle}{1}}{%
    \sectionformat{\chapter}{% Параметры заголовков разделов в тексте
        label=\chaptername\ \thechapter\cftchapaftersnum,
        labelsep=0em,
    }
    %% Следующие две строки: будет вписано слово Глава перед каждым номером раздела в оглавлении   
    \renewcommand{\cftchappresnum}{\chaptername\ }
    \setlength{\cftchapnumwidth}{\widthof{\cftchapfont\cftchappresnum\thechapter\cftchapaftersnum}}
}%

%% Интервалы между заголовками
% На эти величины titlespacing множит через *
\beforetitleunit=\curtextsize% привязались к нашему размеру шрифта
\aftertitleunit=\curtextsize% привязались к нашему размеру шрифта

% Счётчик intvl и длина \otstup определены в файле setup
\titlespacing{\chapter}{\theotstup\parindent}{-1.7em}{*\theintvl}       % Заголовки отделяют от текста сверху и снизу тремя интервалами (ГОСТ Р 7.0.11-2011, 5.3.5). Поскольку название главы всегда является первым на странице, то перед ним отделять пустым тройным интервалом не следует. Независимо от основного шрифта, в этом случае зануление происходит при -1.7em.
\titlespacing{\section}{\theotstup\parindent}{*\theintvl}{*\theintvl}
\titlespacing{\subsection}{\theotstup\parindent}{*\theintvl}{*\theintvl}
\titlespacing{\subsubsection}{\theotstup\parindent}{*\theintvl}{*\theintvl}

%%% Блок дополнительного управления размерами заголовков
\ifthenelse{\equal{\theheadingsize}{1}}{% Пропорциональные заголовки и базовый шрифт 14 пт
    \renewcommand{\cfttoctitlefont}{\hdngaligni\Large\bfseries} % Исправляем размер заголовка оглавления
    \setlength{\cftbeforetoctitleskip}{-1.2\curtextsize}        % Исправляем вертикальный отступ перед заголовком оглавления
    \renewcommand{\cftloftitlefont}{\hdngaligni\Large\bfseries} % Исправляем размер заголовка списка рисунков
    \setlength{\cftbeforeloftitleskip}{-1.4\curtextsize}        % Исправляем вертикальный отступ перед заголовком списка рисунков
    \renewcommand{\cftlottitlefont}{\hdngaligni\Large\bfseries} % Исправляем размер заголовка списка таблиц 
    \setlength{\cftbeforelottitleskip}{-1.4\curtextsize}        % Исправляем вертикальный отступ перед заголовком списка таблиц
    \sectionformat{\chapter}{% Параметры заголовков разделов в тексте
        format=\hdngalign\Large\bfseries, % Исправляем размер заголовка
        top-=0.4em,                       % Исправляем вертикальный отступ перед заголовком
    }
    \sectionformat{\section}{% Параметры заголовков подразделов в тексте
        format=\hdngalign\large\bfseries, % Исправляем размер заголовка
    }
}

\ifthenelse{\equal{\theheadingsize}{1}\AND \curtextsize < \bigtextsize}{% Пропорциональные заголовки и базовый шрифт 14 пт
    \sectionformat{\chapter}{% Параметры заголовков разделов в тексте
        top-=0.2em, % Исправляем вертикальный отступ перед заголовком
    }
}

%%% Счётчики %%%

%% Упрощённые настройки шаблона диссертации: нумерация формул, таблиц, рисунков
\ifthenelse{\equal{\thecontnumeq}{1}}{%
    \counterwithout{equation}{chapter} % Убираем связанность номера формулы с номером главы/раздела
}
\ifthenelse{\equal{\thecontnumfig}{1}}{%
    \counterwithout{figure}{chapter}   % Убираем связанность номера рисунка с номером главы/раздела
}
\ifthenelse{\equal{\thecontnumtab}{1}}{%
    \counterwithout{table}{chapter}    % Убираем связанность номера таблицы с номером главы/раздела
}


%%http://www.linux.org.ru/forum/general/6993203#comment-6994589 (используется totcount)
\makeatletter
\def\formbytotal#1#2#3#4#5{%
    \newcount\@c
    \@c\totvalue{#1}\relax
    \newcount\@last
    \newcount\@pnul
    \@last\@c\relax
    \divide\@last 10
    \@pnul\@last\relax
    \divide\@pnul 10
    \multiply\@pnul-10
    \advance\@pnul\@last
    \multiply\@last-10
    \advance\@last\@c
    \total{#1}~#2%
    \ifnum\@pnul=1#5\else%
    \ifcase\@last#5\or#3\or#4\or#4\or#4\else#5\fi
    \fi
}
\makeatother

\AtBeginDocument{
%% регистрируем счётчики в системе totcounter
    \regtotcounter{totalcount@figure}
    \regtotcounter{totalcount@table}       % Если иным способом поставить в преамбуле то ошибка в числе таблиц
    \regtotcounter{TotPages}               % Если иным способом поставить в преамбуле то ошибка в числе страниц
}
%%%%%%%%%%%%%%%%%%%%%%%%%%%%%%%%%%%%%%%%%%%%%%%%%%%%%%%%%%%%%%%%%%%%%%%%%%%
%%%%%%%%%%%%%%%%%%%%%%%%%%%%%%%%%%%%%%%%%%%%%%%%%%%%%%%%%%%%%%%%%%%%%%%%%%%
% для вертикального центрирования ячеек в tabulary
\def\zz{\ifx\[$\else\aftergroup\zzz\fi}
\def\zzz{\setbox0\lastbox
\dimen0\dimexpr\extrarowheight + \ht0-\dp0\relax
\setbox0\hbox{\raise-.5\dimen0\box0}%
\ht0=\dimexpr\ht0+\extrarowheight\relax
\dp0=\dimexpr\dp0+\extrarowheight\relax 
\box0
}



\lstdefinelanguage{Renhanced}%
{keywords={abbreviate,abline,abs,acos,acosh,action,add1,add,%
        aggregate,alias,Alias,alist,all,anova,any,aov,aperm,append,apply,%
        approx,approxfun,apropos,Arg,args,array,arrows,as,asin,asinh,%
        atan,atan2,atanh,attach,attr,attributes,autoload,autoloader,ave,%
        axis,backsolve,barplot,basename,besselI,besselJ,besselK,besselY,%
        beta,binomial,body,box,boxplot,break,browser,bug,builtins,bxp,by,%
        c,C,call,Call,case,cat,category,cbind,ceiling,character,char,%
        charmatch,check,chol,chol2inv,choose,chull,class,close,cm,codes,%
        coef,coefficients,co,col,colnames,colors,colours,commandArgs,%
        comment,complete,complex,conflicts,Conj,contents,contour,%
        contrasts,contr,control,helmert,contrib,convolve,cooks,coords,%
        distance,coplot,cor,cos,cosh,count,fields,cov,covratio,wt,CRAN,%
        create,crossprod,cummax,cummin,cumprod,cumsum,curve,cut,cycle,D,%
        data,dataentry,date,dbeta,dbinom,dcauchy,dchisq,de,debug,%
        debugger,Defunct,default,delay,delete,deltat,demo,de,density,%
        deparse,dependencies,Deprecated,deriv,description,detach,%
        dev2bitmap,dev,cur,deviance,off,prev,,dexp,df,dfbetas,dffits,%
        dgamma,dgeom,dget,dhyper,diag,diff,digamma,dim,dimnames,dir,%
        dirname,dlnorm,dlogis,dnbinom,dnchisq,dnorm,do,dotplot,double,%
        download,dpois,dput,drop,drop1,dsignrank,dt,dummy,dump,dunif,%
        duplicated,dweibull,dwilcox,dyn,edit,eff,effects,eigen,else,%
        emacs,end,environment,env,erase,eval,equal,evalq,example,exists,%
        exit,exp,expand,expression,External,extract,extractAIC,factor,%
        fail,family,fft,file,filled,find,fitted,fivenum,fix,floor,for,%
        For,formals,format,formatC,formula,Fortran,forwardsolve,frame,%
        frequency,ftable,ftable2table,function,gamma,Gamma,gammaCody,%
        gaussian,gc,gcinfo,gctorture,get,getenv,geterrmessage,getOption,%
        getwd,gl,glm,globalenv,gnome,GNOME,graphics,gray,grep,grey,grid,%
        gsub,hasTsp,hat,heat,help,hist,home,hsv,httpclient,I,identify,if,%
        ifelse,Im,image,\%in\%,index,influence,measures,inherits,install,%
        installed,integer,interaction,interactive,Internal,intersect,%
        inverse,invisible,IQR,is,jitter,kappa,kronecker,labels,lapply,%
        layout,lbeta,lchoose,lcm,legend,length,levels,lgamma,library,%
        licence,license,lines,list,lm,load,local,locator,log,log10,log1p,%
        log2,logical,loglin,lower,lowess,ls,lsfit,lsf,ls,machine,Machine,%
        mad,mahalanobis,make,link,margin,match,Math,matlines,mat,matplot,%
        matpoints,matrix,max,mean,median,memory,menu,merge,methods,min,%
        missing,Mod,mode,model,response,mosaicplot,mtext,mvfft,na,nan,%
        names,omit,nargs,nchar,ncol,NCOL,new,next,NextMethod,nextn,%
        nlevels,nlm,noquote,NotYetImplemented,NotYetUsed,nrow,NROW,null,%
        numeric,\%o\%,objects,offset,old,on,Ops,optim,optimise,optimize,%
        options,or,order,ordered,outer,package,packages,page,pairlist,%
        pairs,palette,panel,par,parent,parse,paste,path,pbeta,pbinom,%
        pcauchy,pchisq,pentagamma,persp,pexp,pf,pgamma,pgeom,phyper,pico,%
        pictex,piechart,Platform,plnorm,plogis,plot,pmatch,pmax,pmin,%
        pnbinom,pnchisq,pnorm,points,poisson,poly,polygon,polyroot,pos,%
        postscript,power,ppoints,ppois,predict,preplot,pretty,Primitive,%
        print,prmatrix,proc,prod,profile,proj,prompt,prop,provide,%
        psignrank,ps,pt,ptukey,punif,pweibull,pwilcox,q,qbeta,qbinom,%
        qcauchy,qchisq,qexp,qf,qgamma,qgeom,qhyper,qlnorm,qlogis,qnbinom,%
        qnchisq,qnorm,qpois,qqline,qqnorm,qqplot,qr,Q,qty,qy,qsignrank,%
        qt,qtukey,quantile,quasi,quit,qunif,quote,qweibull,qwilcox,%
        rainbow,range,rank,rbeta,rbind,rbinom,rcauchy,rchisq,Re,read,csv,%
        csv2,fwf,readline,socket,real,Recall,rect,reformulate,regexpr,%
        relevel,remove,rep,repeat,replace,replications,report,require,%
        resid,residuals,restart,return,rev,rexp,rf,rgamma,rgb,rgeom,R,%
        rhyper,rle,rlnorm,rlogis,rm,rnbinom,RNGkind,rnorm,round,row,%
        rownames,rowsum,rpois,rsignrank,rstandard,rstudent,rt,rug,runif,%
        rweibull,rwilcox,sample,sapply,save,scale,scan,scan,screen,sd,se,%
        search,searchpaths,segments,seq,sequence,setdiff,setequal,set,%
        setwd,show,sign,signif,sin,single,sinh,sink,solve,sort,source,%
        spline,splinefun,split,sqrt,stars,start,stat,stem,step,stop,%
        storage,strstrheight,stripplot,strsplit,structure,strwidth,sub,%
        subset,substitute,substr,substring,sum,summary,sunflowerplot,svd,%
        sweep,switch,symbol,symbols,symnum,sys,status,system,t,table,%
        tabulate,tan,tanh,tapply,tempfile,terms,terrain,tetragamma,text,%
        time,title,topo,trace,traceback,transform,tri,trigamma,trunc,try,%
        ts,tsp,typeof,unclass,undebug,undoc,union,unique,uniroot,unix,%
        unlink,unlist,unname,untrace,update,upper,url,UseMethod,var,%
        variable,vector,Version,vi,warning,warnings,weighted,weights,%
        which,while,window,write,\%x\%,x11,X11,xedit,xemacs,xinch,xor,%
        xpdrows,xy,xyinch,yinch,zapsmall,zip},%
    otherkeywords={!,!=,~,$,*,\%,\&,\%/\%,\%*\%,\%\%,<-,<<-},%
    alsoother={._$},%
    sensitive,%
    morecomment=[l]\#,%
    morestring=[d]",%
    morestring=[d]'% 2001 Robert Denham
}%

%решаем проблему с кириллицей в комментариях (в pdflatex) https://tex.stackexchange.com/a/103712/79756
\lstset{extendedchars=true,literate={Ö}{{\"O}}1
    {Ä}{{\"A}}1
    {Ü}{{\"U}}1
    {ß}{{\ss}}1
    {ü}{{\"u}}1
    {ä}{{\"a}}1
    {ö}{{\"o}}1
    {~}{{\textasciitilde}}1
    {а}{{\selectfont\char224}}1
    {б}{{\selectfont\char225}}1
    {в}{{\selectfont\char226}}1
    {г}{{\selectfont\char227}}1
    {д}{{\selectfont\char228}}1
    {е}{{\selectfont\char229}}1
    {ё}{{\"e}}1
    {ж}{{\selectfont\char230}}1
    {з}{{\selectfont\char231}}1
    {и}{{\selectfont\char232}}1
    {й}{{\selectfont\char233}}1
    {к}{{\selectfont\char234}}1
    {л}{{\selectfont\char235}}1
    {м}{{\selectfont\char236}}1
    {н}{{\selectfont\char237}}1
    {о}{{\selectfont\char238}}1
    {п}{{\selectfont\char239}}1
    {р}{{\selectfont\char240}}1
    {с}{{\selectfont\char241}}1
    {т}{{\selectfont\char242}}1
    {у}{{\selectfont\char243}}1
    {ф}{{\selectfont\char244}}1
    {х}{{\selectfont\char245}}1
    {ц}{{\selectfont\char246}}1
    {ч}{{\selectfont\char247}}1
    {ш}{{\selectfont\char248}}1
    {щ}{{\selectfont\char249}}1
    {ъ}{{\selectfont\char250}}1
    {ы}{{\selectfont\char251}}1
    {ь}{{\selectfont\char252}}1
    {э}{{\selectfont\char253}}1
    {ю}{{\selectfont\char254}}1
    {я}{{\selectfont\char255}}1
    {А}{{\selectfont\char192}}1
    {Б}{{\selectfont\char193}}1
    {В}{{\selectfont\char194}}1
    {Г}{{\selectfont\char195}}1
    {Д}{{\selectfont\char196}}1
    {Е}{{\selectfont\char197}}1
    {Ё}{{\"E}}1
    {Ж}{{\selectfont\char198}}1
    {З}{{\selectfont\char199}}1
    {И}{{\selectfont\char200}}1
    {Й}{{\selectfont\char201}}1
    {К}{{\selectfont\char202}}1
    {Л}{{\selectfont\char203}}1
    {М}{{\selectfont\char204}}1
    {Н}{{\selectfont\char205}}1
    {О}{{\selectfont\char206}}1
    {П}{{\selectfont\char207}}1
    {Р}{{\selectfont\char208}}1
    {С}{{\selectfont\char209}}1
    {Т}{{\selectfont\char210}}1
    {У}{{\selectfont\char211}}1
    {Ф}{{\selectfont\char212}}1
    {Х}{{\selectfont\char213}}1
    {Ц}{{\selectfont\char214}}1
    {Ч}{{\selectfont\char215}}1
    {Ш}{{\selectfont\char216}}1
    {Щ}{{\selectfont\char217}}1
    {Ъ}{{\selectfont\char218}}1
    {Ы}{{\selectfont\char219}}1
    {Ь}{{\selectfont\char220}}1
    {Э}{{\selectfont\char221}}1
    {Ю}{{\selectfont\char222}}1
    {Я}{{\selectfont\char223}}1
    {і}{{\selectfont\char105}}1
    {ї}{{\selectfont\char168}}1
    {є}{{\selectfont\char185}}1
    {ґ}{{\selectfont\char160}}1
    {І}{{\selectfont\char73}}1
    {Ї}{{\selectfont\char136}}1
    {Є}{{\selectfont\char153}}1
    {Ґ}{{\selectfont\char128}}1
}

% Ширина текста минус ширина надписи 999
\newlength{\twless}
\newlength{\lmarg}
\setlength{\lmarg}{\widthof{999}}   % ширина надписи 999
\setlength{\twless}{\textwidth-\lmarg}


\lstset{ %
%    language=R,                     %  Язык указать здесь, если во всех листингах преимущественно один язык, в результате часть настроек может пойти только для этого языка
    numbers=left,                   % where to put the line-numbers
    numberstyle=\fontsize{12pt}{14pt}\selectfont\color{Gray},  % the style that is used for the line-numbers
    firstnumber=2,                  % в этой и следующей строках задаётся поведение нумерации 5, 10, 15...
    stepnumber=5,                   % the step between two line-numbers. If it's 1, each line will be numbered
    numbersep=5pt,                  % how far the line-numbers are from the code
    backgroundcolor=\color{white},  % choose the background color. You must add \usepackage{color}
    showspaces=false,               % show spaces adding particular underscores
    showstringspaces=false,         % underline spaces within strings
    showtabs=false,                 % show tabs within strings adding particular underscores
    frame=leftline,                 % adds a frame of different types around the code
    rulecolor=\color{black},        % if not set, the frame-color may be changed on line-breaks within not-black text (e.g. commens (green here))
    tabsize=2,                      % sets default tabsize to 2 spaces
    captionpos=t,                   % sets the caption-position to top
    breaklines=true,                % sets automatic line breaking
    breakatwhitespace=false,        % sets if automatic breaks should only happen at whitespace
%    title=\lstname,                 % show the filename of files included with \lstinputlisting;
    % also try caption instead of title
    basicstyle=\fontsize{12pt}{14pt}\selectfont\ttfamily,% the size of the fonts that are used for the code
%    keywordstyle=\color{blue},      % keyword style
    commentstyle=\color{ForestGreen}\emph,% comment style
    stringstyle=\color{Mahogany},   % string literal style
    escapeinside={\%*}{*)},         % if you want to add a comment within your code
    morekeywords={*,...},           % if you want to add more keywords to the set
    inputencoding=utf8,             % кодировка кода
    xleftmargin={\lmarg},           % Чтобы весь код и полоска с номерами строк была смещена влево, так чтобы цифры не вылезали за пределы текста слева
} 

%http://tex.stackexchange.com/questions/26872/smaller-frame-with-listings
% Окружение, чтобы листинг был компактнее обведен рамкой, если она задается, а не на всю ширину текста
\makeatletter
\newenvironment{SmallListing}[1][]
{\lstset{#1}\VerbatimEnvironment\begin{VerbatimOut}{VerbEnv.tmp}}
{\end{VerbatimOut}\settowidth\@tempdima{%
        \lstinputlisting{VerbEnv.tmp}}
    \minipage{\@tempdima}\lstinputlisting{VerbEnv.tmp}\endminipage}    
\makeatother


\DefineVerbatimEnvironment% с шрифтом 12 пт
{Verb}{Verbatim}
{fontsize=\fontsize{12pt}{14pt}\selectfont}

\RawFloats[figure,table]            % Отмена установок пакета floatrow для всех флотов (плавающих окружений) выбранных типов или подтипов. А то будто мы зря задавали настройки подписей рисунков и таблиц. 

\DeclareNewFloatType{ListingEnv}{
    placement=htb,
    within=chapter,
    fileext=lol,
    name=Листинг,
}

\captionsetup[ListingEnv]{
    format=tablecaption,
    labelsep=space,                 % Точка после номера листинга задается значением period
    singlelinecheck=off,
    font=onehalfspacing,
    position=top,
}


\floatsetup[ListingEnv]{
    style=plaintop,
    captionskip=4pt,
}

\captionsetup[lstlisting]{
    format=tablecaption,
    labelsep=space,                 % Точка после номера листинга задается значением period
    singlelinecheck=off,
    font=onehalfspacing,
    position=top,
}

\renewcommand{\lstlistingname}{Листинг}

%Общие счётчики окружений листингов
%http://tex.stackexchange.com/questions/145546/how-to-make-figure-and-listing-share-their-counter
% Если смешивать плавающие и не плавающие окружения, то могут быть проблемы с нумерацией
\makeatletter
\AtBeginDocument{%
    \let\c@ListingEnv\c@lstlisting
    \let\theListingEnv\thelstlisting
    \let\ftype@lstlisting\ftype@ListingEnv % give the floats the same precedence
}
\makeatother

% значок С++ — используйте команду \cpp
\newcommand{\cpp}{%
    C\nolinebreak\hspace{-.05em}%
    \raisebox{.2ex}{+}\nolinebreak\hspace{-.10em}%
    \raisebox{.2ex}{+}%
}


%%% Русская традиция начертания математических знаков
%\renewcommand{\le}{\ensuremath{\leqslant}}
%\renewcommand{\leq}{\ensuremath{\leqslant}}
%\renewcommand{\ge}{\ensuremath{\geqslant}}
%\renewcommand{\geq}{\ensuremath{\geqslant}}
%\renewcommand{\emptyset}{\varnothing}

%%% Русская традиция начертания греческих букв (греческие буквы вертикальные, через пакет upgreek)
%\renewcommand{\epsilon}{\ensuremath{\upvarepsilon}}   %  русская традиция записи
%\renewcommand{\phi}{\ensuremath{\upvarphi}}
%%\renewcommand{\kappa}{\ensuremath{\varkappa}}
%\renewcommand{\alpha}{\upalpha}
%\renewcommand{\beta}{\upbeta}
%\renewcommand{\gamma}{\upgamma}
%\renewcommand{\delta}{\updelta}
%\renewcommand{\varepsilon}{\upvarepsilon}
%\renewcommand{\zeta}{\upzeta}
%\renewcommand{\eta}{\upeta}
%\renewcommand{\theta}{\uptheta}
%\renewcommand{\vartheta}{\upvartheta}
%\renewcommand{\iota}{\upiota}
%\renewcommand{\kappa}{\upkappa}
%\renewcommand{\lambda}{\uplambda}
%\renewcommand{\mu}{\upmu}
%\renewcommand{\nu}{\upnu}
%\renewcommand{\xi}{\upxi}
%\renewcommand{\pi}{\uppi}
%\renewcommand{\varpi}{\upvarpi}
%\renewcommand{\rho}{\uprho}
%%\renewcommand{\varrho}{\upvarrho}
%\renewcommand{\sigma}{\upsigma}
%%\renewcommand{\varsigma}{\upvarsigma}
%\renewcommand{\tau}{\uptau}
%\renewcommand{\upsilon}{\upupsilon}
%\renewcommand{\varphi}{\upvarphi}
%\renewcommand{\chi}{\upchi}
%\renewcommand{\psi}{\uppsi}
%\renewcommand{\omega}{\upomega}
          % Стили для специфических пользовательских задач
%%%%%%%%%%%%%%%%%%%%%%%%%%%%%%%%%%%%%%%%%%%%%%%%%%%%%%%%%%%%%%%%%%%%%%%%%%%
%%%%%%%%%%%%%%%%%%%%%%%%%%%%%%%%%%%%%%%%%%%%%%%%%%%%%%%%%%%%%%%%%%%%%%%%%%%
%%%% Библиография. Общие настройки для двух способов её подключения %%%


%%% Выбор реализации %%%
\ifthenelse{\equal{\thebibliosel}{0}}{%
    %%% Реализация библиографии встроенными средствами посредством движка bibtex8 %%%

%%% Пакеты %%%
\usepackage{cite}                                   % Красивые ссылки на литературу


%%% Стили %%%
\bibliographystyle{BibTeX-Styles/utf8gost71u}    % Оформляем библиографию по ГОСТ 7.1 (ГОСТ Р 7.0.11-2011, 5.6.7)

\makeatletter
\renewcommand{\@biblabel}[1]{#1.}   % Заменяем библиографию с квадратных скобок на точку
\makeatother
%% Управление отступами между записями
%% требует etoolbox 
%% http://tex.stackexchange.com/a/105642
%\patchcmd\thebibliography
% {\labelsep}
% {\labelsep\itemsep=5pt\parsep=0pt\relax}
% {}
% {\typeout{Couldn't patch the command}}

%%% Список литературы с красной строки (без висячего отступа) %%%
%\patchcmd{\thebibliography} %может потребовать включения пакета etoolbox
%  {\advance\leftmargin\labelsep}
%  {\leftmargin=0pt%
%   \setlength{\labelsep}{\widthof{\ }}% Управляет длиной отступа после точки
%   \itemindent=\parindent%
%   \addtolength{\itemindent}{\labelwidth}% Сдвигаем правее на величину номера с точкой
%   \advance\itemindent\labelsep%
%  }
%  {}{}

%%% Цитирование %%%
\renewcommand\citepunct{;\penalty\citepunctpenalty%
    \hskip.13emplus.1emminus.1em\relax}                % Разделение ; при перечислении ссылок (ГОСТ Р 7.0.5-2008)


%%% Создание команд для вывода списка литературы %%%
\newcommand*{\insertbibliofull}{
\bibliography{biblio/othercites,biblio/authorpapersVAK,biblio/authorpapers,biblio/authorconferences}         % Подключаем BibTeX-базы % После запятых не должно быть лишних пробелов — он "думает", что это тоже имя пути
}

\newcommand*{\insertbiblioauthor}{
\bibliography{biblio/authorpapersVAK,biblio/authorpapers,biblio/authorconferences}         % Подключаем BibTeX-базы % После запятых не должно быть лишних пробелов — он "думает", что это тоже имя пути
}

\newcommand*{\insertbiblioother}{
\bibliography{biblio/othercites}         % Подключаем BibTeX-базы
}


%% Счётчик использованных ссылок на литературу, обрабатывающий с учётом неоднократных ссылок
%% Требуется дважды компилировать, поскольку ему нужно считать актуальный внешний файл со списком литературы
\newtotcounter{citenum}
\def\oldcite{}
\let\oldcite=\bibcite
\def\bibcite{\stepcounter{citenum}\oldcite}
  % Встроенная реализация с загрузкой файла через движок bibtex8
}{
    %%% Реализация библиографии пакетами biblatex и biblatex-gost с использованием движка biber %%%

%\usepackage{csquotes} % biblatex рекомендует его подключать. Пакет для оформления сложных блоков цитирования.

%%% Загрузка пакета с основными настройками %%%
\usepackage[%
backend=biber,% движок
bibencoding=utf8,% кодировка bib файла
sorting=none,% настройка сортировки списка литературы
style=gost-numeric,% стиль цитирования и библиографии (по ГОСТ)
language=autobib,% получение языка из babel/polyglossia, default: autobib % если ставить autocite или auto, то цитаты в тексте с указанием страницы, получат указание страницы на языке оригинала
autolang=other,% многоязычная библиография
clearlang=true,% внутренний сброс поля language, если он совпадает с языком из babel/polyglossia
defernumbers=true,% нумерация проставляется после двух компиляций, зато позволяет выцеплять библиографию по ключевым словам и нумеровать не из большего списка
sortcites=true,% сортировать номера затекстовых ссылок при цитировании (если в квадратных скобках несколько ссылок, то отображаться будут отсортированно, а не абы как)
%doi=false,% Показывать или нет ссылки на DOI
%isbn=false,% Показывать или нет ISBN
]{biblatex}



%http://tex.stackexchange.com/a/141831/79756
%There is a way to automatically map the language field to the langid field. The following lines in the preamble should be enough to do that.
%This command will copy the language field into the langid field and will then delete the contents of the language field. The language field will only be deleted if it was successfully copied into the langid field.
\DeclareSourcemap{ %модификация bib файла перед тем, как им займётся biblatex 
    \maps{
        \map{% перекидываем значения полей language в поля langid, которыми пользуется biblatex
            \step[fieldsource=language, fieldset=langid, origfieldval, final]
            \step[fieldset=language, null]
        }
        \map{% перекидываем значения полей numpages в поля pagetotal, которыми пользуется biblatex
            \step[fieldsource=numpages, fieldset=pagetotal, origfieldval, final]
            \step[fieldset=pagestotal, null]
        }
        \map{% если в поле medium написано "Электронный ресурс", то устанавливаем поле media. которым пользуется biblatex в значение eresource
            \step[fieldsource=medium,
            match=\regexp{Электронный\s+ресурс},
            final]
            \step[fieldset=media, fieldvalue=eresource]
        }
        \map[overwrite]{% стираем значения всех полей issn
            \step[fieldset=issn, null]
        }
        \map[overwrite]{% стираем значения всех полей abstract, поскольку ими не пользуемся, а там бывают "неприятные" латеху символы
            \step[fieldsource=abstract]
            \step[fieldset=abstract,null]
        }
        \map[overwrite]{ % переделка формата записи даты
            \step[fieldsource=urldate,
            match=\regexp{([0-9]{2})\.([0-9]{2})\.([0-9]{4})},
            replace={$3-$2-$1$4}, % $4 вставлен исключительно ради нормальной работы программ подсветки синтаксиса, которые некорректно обрабатывают $ в таких конструкциях
            final]
        }
        \map[overwrite]{ % добавляем ключевые слова, чтобы различать источники
            \perdatasource{biblio/othercites.bib}
            \step[fieldset=keywords, fieldvalue={biblioother,bibliofull}]
        }
        \map[overwrite]{ % добавляем ключевые слова, чтобы различать источники
            \perdatasource{biblio/authorpapersVAK.bib}
            \step[fieldset=keywords, fieldvalue={biblioauthorvak,biblioauthor,bibliofull}]
        }
        \map[overwrite]{ % добавляем ключевые слова, чтобы различать источники
            \perdatasource{biblio/authorpapers.bib}
            \step[fieldset=keywords, fieldvalue={biblioauthornotvak,biblioauthor,bibliofull}]
        }
        \map[overwrite]{ % добавляем ключевые слова, чтобы различать источники
            \perdatasource{biblio/authorconferences.bib}
            \step[fieldset=keywords, fieldvalue={biblioauthorconf,biblioauthor,bibliofull}]
        }
%        \map[overwrite]{% стираем значения всех полей series
%            \step[fieldset=series, null]
%        }
        \map[overwrite]{% перекидываем значения полей howpublished в поля organization для типа online
            \step[typesource=online, fieldsource=howpublished, fieldset=organization, origfieldval, final]
            \step[fieldset=howpublished, null]
        }
        % Так отключаем [Электронный ресурс]
%        \map[overwrite]{% стираем значения всех полей media=eresource
%            \step[fieldsource=media,
%            match={eresource},
%            final]
%            \step[fieldset=media, null]
%        }
    }
}

%%% Правка записей типа thesis, чтобы дважды не писался автор
%\DeclareBibliographyDriver{thesis}{%
%  \usebibmacro{bibindex}%
%  \usebibmacro{begentry}%
%  \usebibmacro{heading}%
%  \newunit
%  \usebibmacro{author}%
%  \setunit*{\labelnamepunct}%
%  \usebibmacro{thesistitle}%
%  \setunit{\respdelim}%
%  %\printnames[last-first:full]{author}%Вот эту строчку нужно убрать, чтобы автор диссертации не дублировался
%  \newunit\newblock
%  \printlist[semicolondelim]{specdata}%
%  \newunit
%  \usebibmacro{institution+location+date}%
%  \newunit\newblock
%  \usebibmacro{chapter+pages}%
%  \newunit
%  \printfield{pagetotal}%
%  \newunit\newblock
%  \usebibmacro{doi+eprint+url+note}%
%  \newunit\newblock
%  \usebibmacro{addendum+pubstate}%
%  \setunit{\bibpagerefpunct}\newblock
%  \usebibmacro{pageref}%
%  \newunit\newblock
%  \usebibmacro{related:init}%
%  \usebibmacro{related}%
%  \usebibmacro{finentry}}


%\newbibmacro{string+doi}[1]{% новая макрокоманда на простановку ссылки на doi
%    \iffieldundef{doi}{#1}{\href{http://dx.doi.org/\thefield{doi}}{#1}}}
%
%\renewcommand*{\mkgostheading}[1]{\usebibmacro{string+doi}{#1}} % ссылка на doi с авторов. стоящих впереди записи
%\renewcommand*{\mkgostheading}[1]{#1} % только лишь убираем курсив с авторов
%\DeclareFieldFormat{title}{\usebibmacro{string+doi}{#1}} % ссылка на doi с названия работы
%\DeclareFieldFormat{journaltitle}{\usebibmacro{string+doi}{#1}} % ссылка на doi с названия журнала
% Убрать тире из разделителей элементов в библиографии:
%\renewcommand*{\newblockpunct}{%
%    \addperiod\space\bibsentence}%block punct.,\bibsentence is for vol,etc.

%%% Возвращаем запись «Режим доступа» %%%
%\DefineBibliographyStrings{english}{%
%    urlfrom = {Mode of access}
%}
%\DeclareFieldFormat{url}{\bibstring{urlfrom}\addcolon\space\url{#1}}

%%% Set low penalties for breaks at uppercase letters and lowercase letters
%\setcounter{biburllcpenalty}{500} %управляет разрывами ссылок после маленьких букв RTFM biburllcpenalty
%\setcounter{biburlucpenalty}{3000} %управляет разрывами ссылок после больших букв, RTFM biburlucpenalty

%%% Список литературы с красной строки (без висячего отступа) %%%
%\defbibenvironment{bibliography} % переопределяем окружение библиографии из gost-numeric.bbx пакета biblatex-gost
%  {\list
%     {\printtext[labelnumberwidth]{%
%	\printfield{prefixnumber}%
%	\printfield{labelnumber}}}
%     {%
%      \setlength{\labelwidth}{\labelnumberwidth}%
%      \setlength{\leftmargin}{0pt}% default is \labelwidth
%      \setlength{\labelsep}{\widthof{\ }}% Управляет длиной отступа после точки % default is \biblabelsep
%      \setlength{\itemsep}{\bibitemsep}% Управление дополнительным вертикальным разрывом между записями. \bibitemsep по умолчанию соответствует \itemsep списков в документе.
%      \setlength{\itemindent}{\bibhang}% Пользуемся тем, что \bibhang по умолчанию принимает значение \parindent (абзацного отступа), который переназначен в styles.tex
%      \addtolength{\itemindent}{\labelwidth}% Сдвигаем правее на величину номера с точкой
%      \addtolength{\itemindent}{\labelsep}% Сдвигаем ещё правее на отступ после точки
%      \setlength{\parsep}{\bibparsep}%
%     }%
%      \renewcommand*{\makelabel}[1]{\hss##1}%
%  }
%  {\endlist}
%  {\item}

%%% Подключение файлов bib %%%
\addbibresource{biblio/othercites.bib}
\addbibresource{biblio/authorpapersVAK.bib}
\addbibresource{biblio/authorpapers.bib}
\addbibresource{biblio/authorconferences.bib}


%% Счётчик использованных ссылок на литературу, обрабатывающий с учётом неоднократных ссылок
%http://tex.stackexchange.com/a/66851/79756
%\newcounter{citenum}
\newtotcounter{citenum}
\makeatletter
\defbibenvironment{counter} %Env of bibliography
  {\setcounter{citenum}{0}%
  \renewcommand{\blx@driver}[1]{}%
  } %what is doing at the beginining of bibliography. In your case it's : a. Reset counter b. Say to print nothing when a entry is tested.
  {} %Здесь то, что будет выводиться командой \printbibliography. \thecitenum сюда писать не надо
  {\stepcounter{citenum}} %What is printing / executed at each entry.
\makeatother
\defbibheading{counter}{}



\newtotcounter{citeauthorvak}
\makeatletter
\defbibenvironment{countauthorvak} %Env of bibliography
{\setcounter{citeauthorvak}{0}%
    \renewcommand{\blx@driver}[1]{}%
} %what is doing at the beginining of bibliography. In your case it's : a. Reset counter b. Say to print nothing when a entry is tested.
{} %Здесь то, что будет выводиться командой \printbibliography. Обойдёмся без \theciteauthorvak в нашей реализации
{\stepcounter{citeauthorvak}} %What is printing / executed at each entry.
\makeatother
\defbibheading{countauthorvak}{}

\newtotcounter{citeauthornotvak}
\makeatletter
\defbibenvironment{countauthornotvak} %Env of bibliography
{\setcounter{citeauthornotvak}{0}%
    \renewcommand{\blx@driver}[1]{}%
} %what is doing at the beginining of bibliography. In your case it's : a. Reset counter b. Say to print nothing when a entry is tested.
{} %Здесь то, что будет выводиться командой \printbibliography. Обойдёмся без \theciteauthornotvak в нашей реализации
{\stepcounter{citeauthornotvak}} %What is printing / executed at each entry.
\makeatother
\defbibheading{countauthornotvak}{}

\newtotcounter{citeauthorconf}
\makeatletter
\defbibenvironment{countauthorconf} %Env of bibliography
{\setcounter{citeauthorconf}{0}%
    \renewcommand{\blx@driver}[1]{}%
} %what is doing at the beginining of bibliography. In your case it's : a. Reset counter b. Say to print nothing when a entry is tested.
{} %Здесь то, что будет выводиться командой \printbibliography. Обойдёмся без \theciteauthorconf в нашей реализации
{\stepcounter{citeauthorconf}} %What is printing / executed at each entry.
\makeatother
\defbibheading{countauthorconf}{}

\newtotcounter{citeauthor}
\makeatletter
\defbibenvironment{countauthor} %Env of bibliography
{\setcounter{citeauthor}{0}%
    \renewcommand{\blx@driver}[1]{}%
} %what is doing at the beginining of bibliography. In your case it's : a. Reset counter b. Say to print nothing when a entry is tested.
{} %Здесь то, что будет выводиться командой \printbibliography. Обойдёмся без \theciteauthor в нашей реализации
{\stepcounter{citeauthor}} %What is printing / executed at each entry.
\makeatother
\defbibheading{countauthor}{}





%%% Создание команд для вывода списка литературы %%%
\newcommand*{\insertbibliofull}{
\printbibliography[keyword=bibliofull,section=0]
\printbibliography[heading=counter,env=counter,keyword=bibliofull,section=0]
}

\newcommand*{\insertbiblioauthor}{
\printbibliography[keyword=biblioauthor,section=1,title=\authorbibtitle]
\printbibliography[heading=counter,env=counter,keyword=biblioauthor,section=1]
}

\newcommand*{\insertbiblioother}{
\printbibliography[keyword=biblioother]
\printbibliography[heading=counter,env=counter,keyword=biblioother]
}

    % Реализация пакетом biblatex через движок biber
}
% Настройки библиографии из внешнего
%файла (там же выбор: встроенная или на основе biblatex)

%%% Реализация библиографии пакетами biblatex и biblatex-gost с использованием движка biber %%%

%\usepackage{csquotes} % biblatex рекомендует его подключать. Пакет для оформления сложных блоков цитирования.

%%% Загрузка пакета с основными настройками %%%
\usepackage[%
backend=biber,% движок
bibencoding=utf8,% кодировка bib файла
sorting=none,% настройка сортировки списка литературы
% style=gost-numeric,% стиль цитирования и библиографии (по ГОСТ)
% language=autobib,% получение языка из babel/polyglossia, default: autobib % если ставить autocite или auto, то цитаты в тексте с указанием страницы, получат указание страницы на языке оригинала
% autolang=other,% многоязычная библиография
% clearlang=true,% внутренний сброс поля language, если он совпадает с языком из babel/polyglossia
% defernumbers=true,% нумерация проставляется после двух компиляций, зато позволяет выцеплять библиографию по ключевым словам и нумеровать не из большего списка
% sortcites=true,% сортировать номера затекстовых ссылок при цитировании (если в квадратных скобках несколько ссылок, то отображаться будут отсортированно, а не абы как)
% %doi=false,% Показывать или нет ссылки на DOI
% %isbn=false,% Показывать или нет ISBN
]{biblatex}



%http://tex.stackexchange.com/a/141831/79756
%There is a way to automatically map the language field to the langid field. The following lines in the preamble should be enough to do that.
%This command will copy the language field into the langid field and will then delete the contents of the language field. The language field will only be deleted if it was successfully copied into the langid field.
\DeclareSourcemap{ %модификация bib файла перед тем, как им займётся biblatex 
    \maps{
        \map{% перекидываем значения полей language в поля langid, которыми пользуется biblatex
            \step[fieldsource=language, fieldset=langid, origfieldval, final]
            \step[fieldset=language, null]
        }
        \map{% перекидываем значения полей numpages в поля pagetotal, которыми пользуется biblatex
            \step[fieldsource=numpages, fieldset=pagetotal, origfieldval, final]
            \step[fieldset=pagestotal, null]
        }
        \map{% если в поле medium написано "Электронный ресурс", то устанавливаем поле media. которым пользуется biblatex в значение eresource
            \step[fieldsource=medium,
            match=\regexp{Электронный\s+ресурс},
            final]
            \step[fieldset=media, fieldvalue=eresource]
        }
        \map[overwrite]{% стираем значения всех полей issn
            \step[fieldset=issn, null]
        }
        \map[overwrite]{% стираем значения всех полей abstract, поскольку ими не пользуемся, а там бывают "неприятные" латеху символы
            \step[fieldsource=abstract]
            \step[fieldset=abstract,null]
        }
        \map[overwrite]{ % переделка формата записи даты
            \step[fieldsource=urldate,
            match=\regexp{([0-9]{2})\.([0-9]{2})\.([0-9]{4})},
            replace={$3-$2-$1$4}, % $4 вставлен исключительно ради нормальной работы программ подсветки синтаксиса, которые некорректно обрабатывают $ в таких конструкциях
            final]
        }
        \map[overwrite]{ % добавляем ключевые слова, чтобы различать источники
            \perdatasource{../biblio/othercites.bib}
            \step[fieldset=keywords, fieldvalue={biblioother,bibliofull}]
        }
        \map[overwrite]{ % добавляем ключевые слова, чтобы различать источники
            \perdatasource{../biblio/authorpapersVAK.bib}
            \step[fieldset=keywords, fieldvalue={biblioauthorvak,biblioauthor,bibliofull}]
        }
        \map[overwrite]{ % добавляем ключевые слова, чтобы различать источники
            \perdatasource{../biblio/authorpapers.bib}
            \step[fieldset=keywords, fieldvalue={biblioauthornotvak,biblioauthor,bibliofull}]
        }
        \map[overwrite]{ % добавляем ключевые слова, чтобы различать источники
            \perdatasource{../biblio/authorconferences.bib}
            \step[fieldset=keywords, fieldvalue={biblioauthorconf,biblioauthor,bibliofull}]
        }
%        \map[overwrite]{% стираем значения всех полей series
%            \step[fieldset=series, null]
%        }
        \map[overwrite]{% перекидываем значения полей howpublished в поля organization для типа online
            \step[typesource=online, fieldsource=howpublished, fieldset=organization, origfieldval, final]
            \step[fieldset=howpublished, null]
        }
        % Так отключаем [Электронный ресурс]
%        \map[overwrite]{% стираем значения всех полей media=eresource
%            \step[fieldsource=media,
%            match={eresource},
%            final]
%            \step[fieldset=media, null]
%        }
    }
}

%%% Правка записей типа thesis, чтобы дважды не писался автор
%\DeclareBibliographyDriver{thesis}{%
%  \usebibmacro{bibindex}%
%  \usebibmacro{begentry}%
%  \usebibmacro{heading}%
%  \newunit
%  \usebibmacro{author}%
%  \setunit*{\labelnamepunct}%
%  \usebibmacro{thesistitle}%
%  \setunit{\respdelim}%
%  %\printnames[last-first:full]{author}%Вот эту строчку нужно убрать, чтобы автор диссертации не дублировался
%  \newunit\newblock
%  \printlist[semicolondelim]{specdata}%
%  \newunit
%  \usebibmacro{institution+location+date}%
%  \newunit\newblock
%  \usebibmacro{chapter+pages}%
%  \newunit
%  \printfield{pagetotal}%
%  \newunit\newblock
%  \usebibmacro{doi+eprint+url+note}%
%  \newunit\newblock
%  \usebibmacro{addendum+pubstate}%
%  \setunit{\bibpagerefpunct}\newblock
%  \usebibmacro{pageref}%
%  \newunit\newblock
%  \usebibmacro{related:init}%
%  \usebibmacro{related}%
%  \usebibmacro{finentry}}


%\newbibmacro{string+doi}[1]{% новая макрокоманда на простановку ссылки на doi
%    \iffieldundef{doi}{#1}{\href{http://dx.doi.org/\thefield{doi}}{#1}}}
%
%\renewcommand*{\mkgostheading}[1]{\usebibmacro{string+doi}{#1}} % ссылка на doi с авторов. стоящих впереди записи
%\renewcommand*{\mkgostheading}[1]{#1} % только лишь убираем курсив с авторов
%\DeclareFieldFormat{title}{\usebibmacro{string+doi}{#1}} % ссылка на doi с названия работы
%\DeclareFieldFormat{journaltitle}{\usebibmacro{string+doi}{#1}} % ссылка на doi с названия журнала
% Убрать тире из разделителей элементов в библиографии:
%\renewcommand*{\newblockpunct}{%
%    \addperiod\space\bibsentence}%block punct.,\bibsentence is for vol,etc.

%%% Возвращаем запись «Режим доступа» %%%
%\DefineBibliographyStrings{english}{%
%    urlfrom = {Mode of access}
%}
%\DeclareFieldFormat{url}{\bibstring{urlfrom}\addcolon\space\url{#1}}

%%% Set low penalties for breaks at uppercase letters and lowercase letters
%\setcounter{biburllcpenalty}{500} %управляет разрывами ссылок после маленьких букв RTFM biburllcpenalty
%\setcounter{biburlucpenalty}{3000} %управляет разрывами ссылок после больших букв, RTFM biburlucpenalty

%%% Список литературы с красной строки (без висячего отступа) %%%
%\defbibenvironment{bibliography} % переопределяем окружение библиографии из gost-numeric.bbx пакета biblatex-gost
%  {\list
%     {\printtext[labelnumberwidth]{%
%	\printfield{prefixnumber}%
%	\printfield{labelnumber}}}
%     {%
%      \setlength{\labelwidth}{\labelnumberwidth}%
%      \setlength{\leftmargin}{0pt}% default is \labelwidth
%      \setlength{\labelsep}{\widthof{\ }}% Управляет длиной отступа после точки % default is \biblabelsep
%      \setlength{\itemsep}{\bibitemsep}% Управление дополнительным вертикальным разрывом между записями. \bibitemsep по умолчанию соответствует \itemsep списков в документе.
%      \setlength{\itemindent}{\bibhang}% Пользуемся тем, что \bibhang по умолчанию принимает значение \parindent (абзацного отступа), который переназначен в styles.tex
%      \addtolength{\itemindent}{\labelwidth}% Сдвигаем правее на величину номера с точкой
%      \addtolength{\itemindent}{\labelsep}% Сдвигаем ещё правее на отступ после точки
%      \setlength{\parsep}{\bibparsep}%
%     }%
%      \renewcommand*{\makelabel}[1]{\hss##1}%
%  }
%  {\endlist}
%  {\item}

%%% Подключение файлов bib %%%
\addbibresource{../biblio/othercites.bib}
\addbibresource{../biblio/authorpapersVAK.bib}
\addbibresource{../biblio/authorpapers.bib}
\addbibresource{../biblio/authorconferences.bib}


%% Счётчик использованных ссылок на литературу, обрабатывающий с учётом неоднократных ссылок
%http://tex.stackexchange.com/a/66851/79756
%\newcounter{citenum}
\newtotcounter{citenum}
\makeatletter
\defbibenvironment{counter} %Env of bibliography
  {\setcounter{citenum}{0}%
  \renewcommand{\blx@driver}[1]{}%
  } %what is doing at the beginining of bibliography. In your case it's : a. Reset counter b. Say to print nothing when a entry is tested.
  {} %Здесь то, что будет выводиться командой \printbibliography. \thecitenum сюда писать не надо
  {\stepcounter{citenum}} %What is printing / executed at each entry.
\makeatother
\defbibheading{counter}{}



\newtotcounter{citeauthorvak}
\makeatletter
\defbibenvironment{countauthorvak} %Env of bibliography
{\setcounter{citeauthorvak}{0}%
    \renewcommand{\blx@driver}[1]{}%
} %what is doing at the beginining of bibliography. In your case it's : a. Reset counter b. Say to print nothing when a entry is tested.
{} %Здесь то, что будет выводиться командой \printbibliography. Обойдёмся без \theciteauthorvak в нашей реализации
{\stepcounter{citeauthorvak}} %What is printing / executed at each entry.
\makeatother
\defbibheading{countauthorvak}{}

\newtotcounter{citeauthornotvak}
\makeatletter
\defbibenvironment{countauthornotvak} %Env of bibliography
{\setcounter{citeauthornotvak}{0}%
    \renewcommand{\blx@driver}[1]{}%
} %what is doing at the beginining of bibliography. In your case it's : a. Reset counter b. Say to print nothing when a entry is tested.
{} %Здесь то, что будет выводиться командой \printbibliography. Обойдёмся без \theciteauthornotvak в нашей реализации
{\stepcounter{citeauthornotvak}} %What is printing / executed at each entry.
\makeatother
\defbibheading{countauthornotvak}{}

\newtotcounter{citeauthorconf}
\makeatletter
\defbibenvironment{countauthorconf} %Env of bibliography
{\setcounter{citeauthorconf}{0}%
    \renewcommand{\blx@driver}[1]{}%
} %what is doing at the beginining of bibliography. In your case it's : a. Reset counter b. Say to print nothing when a entry is tested.
{} %Здесь то, что будет выводиться командой \printbibliography. Обойдёмся без \theciteauthorconf в нашей реализации
{\stepcounter{citeauthorconf}} %What is printing / executed at each entry.
\makeatother
\defbibheading{countauthorconf}{}

\newtotcounter{citeauthor}
\makeatletter
\defbibenvironment{countauthor} %Env of bibliography
{\setcounter{citeauthor}{0}%
    \renewcommand{\blx@driver}[1]{}%
} %what is doing at the beginining of bibliography. In your case it's : a. Reset counter b. Say to print nothing when a entry is tested.
{} %Здесь то, что будет выводиться командой \printbibliography. Обойдёмся без \theciteauthor в нашей реализации
{\stepcounter{citeauthor}} %What is printing / executed at each entry.
\makeatother
\defbibheading{countauthor}{}





%%% Создание команд для вывода списка литературы %%%
\newcommand*{\insertbibliofull}{
\printbibliography[keyword=bibliofull,section=0]
\printbibliography[heading=counter,env=counter,keyword=bibliofull,section=0]
}

\newcommand*{\insertbiblioauthor}{
\printbibliography[keyword=biblioauthor,section=1,title=\authorbibtitle]
\printbibliography[heading=counter,env=counter,keyword=biblioauthor,section=1]
}

\newcommand*{\insertbiblioother}{
\printbibliography[keyword=biblioother]
\printbibliography[heading=counter,env=counter,keyword=biblioother]
}



%%%%%%%%%%%%%%%%%%%%%%%%%%%%%%%%%%%%%%%%%%%%%%%%%%%%%%%%%%%%%%%%%%%%%%%%%%%
%%%%%%%%%%%%%%%%%%%%%%%%%%%%%%%%%%%%%%%%%%%%%%%%%%%%%%%%%%%%%%%%%%%%%%%%%%%
%%%% Управление компиляцией отдельных частей диссертации %%%
% Необходимо сначала иметь полностью скомпилированный документ, чтобы все
% промежуточные файлы были в наличии
% Затем, для вывода отдельных частей можно воспользоваться командой \includeonly
% Ниже примеры использования команды:
%
%\includeonly{part2}
%\includeonly{contents,appendix,conclusion}
%
% Если все команды закомментированы, то документ будет выведен в PDF файл полностью    % Управление компиляцией отдельных частей диссертации

%%%%%%%%%%%%%%%%%%%%%%%%%%%%%%%%%%%%%%%%%%%%%%%%%%%%%%%%%%%%%%%%%%%%%%%%%%%
%%%%%%%%%%%%%%%%%%%%%%%%%%%%%%%%%%%%%%%%%%%%%%%%%%%%%%%%%%%%%%%%%%%%%%%%%%%
\usepackage{import} % Продвинутый input/include файлов. Внимательно читать документацию пакета про последовательность перебора путей при возникновении проблем. Особенно если есть файлы с одинаковыми именами в директориях разной вложенности.

\chapter{Система Распознавания Рукописных Текстов}

\section{Предобработка изображения}

Удаление шума, применение методов threshoulding-а
Выделение области интереса

Цель этапа предобработки строк - применить последовательность процедур нормализации к изображениям строк с целью сокращения вариативности внешней стилистики рукописного текста. Некоторые из возможных процедур нормализации описаны в следующих разделах.

Более подробно про текхники предобработки текстового изображения можно почитать в \cite{fujisawa2007robustness, durebrandt2015segmentation}.

\subsection{Выделение текста}
\cite{Louloudis2008, Liwicki2007}

История исследования: \cite{durebrandt2015segmentation}, \cite{fischer2012handwriting}, \cite{Louloudis2008}, \cite{Gatos2006}

Stroke Width Transform \cite{Epshtein2010} хорошо показавший себя метод в поике машинопечатного текста на изображениях.

\subsubsection{Бинаризация}

\cite{Gatos2006}

\subsubsection{Выравнивание строк}
\subsection{Сегментация на строки}

\textbf{Vertical Projection Profile (VPP)}
The VPP is essentially a histogram produced by taking the sum of the pixel intensities in each row, from which the line boundaries can be ex- tracted. \cite{durebrandt2015segmentation}
\textbf{Smearing method}
The rectangle based filtering method blurs the image with a horizontally elongated median filter, followed by binarization using Otsu’s method. Then, in the ideal case, the obtained binary image contains a connected component for each text line. The evaluation on the data-set found that the latter method is more successful. The implemented Smearing method described in section 5.4 takes inspiration from this method, but uses an averaging filter instead of a median filter. \cite{durebrandt2015segmentation}

\textbf{Seam Carving} is performed on each segmented text line, and the gaps between the connected com- ponents are classified as either within or between a word. \cite{durebrandt2015segmentation}

A good overviewof algorithms for text-line extraction is given in \cite{likforman2007text}, \cite{durebrandt2015segmentation, Papavassiliou2010, Moysset2013}.

\subsection{Нормализация сегментированных строк}
\subsection{Сегментация на слова}
\textbf{connected component analysis} 
\subsubsection{Скелетонизация}
Скелетонизация (thinning or skeletonization). Целью данной процедуры является нормализация ширины каждой линии на изображении к ширине 1-ого пикселя, сохраняя при это топологию структуры слова. \cite{suen1994thinning}

\subsubsection{Коррекция базовой линии}
Коррекция базовой линии(baseline correction or skew correction). computing a baseline estimate by interpolating local contour minima (cf. [...]) works well for Roman script, but will be likely to fail for Arabic. \cite{plotz2009markov}

More widely applicable is the implicit estimation of the baseline orientation by maximizing the entropy of the horizontal projection histogram (cf., e.g., [...]). \cite{plotz2009markov}

\subsubsection{Коррекция угла наклона}

The slant angle, i.e., the tilt of individual handwritten characters with respect to the vertical, is compensated by applying a shear transform to the text-line image. \cite{plotz2009markov}

\subsection{Нормализация слов}
\subsubsection{Зонирование}
\subsubsection{Коррекция ширины}

A quite robust method for normalizing the size of Roman script was proposed in \cite{madhvanath1999chaincode} which uses an estimate of average character width.

\subsection{Сегментация на уровне слова}

Входными данными для системы распознавания рукописных слов являются нормализованные изображения слов, а выходными - слова в кодировке понятной компьютеру.

В литературе используется несколько направлений к распознаванию, метод распознавания напрямую влияет на метод сегментации слова. 

\subsubsection{Целостный Подход}

Holistic. Этот подход использует слово целиком, без сегментации его на части. Для распознавания используются следующие признаки: кол-во вертикальных линий, длинна, относительное расстояние между нижними штрихами и верхними штрихами, кол-во тех и друх штрихов, замкнутые области(циклы). Данные подход показывает хорошие разультаты на наборах данных с маленьким словарем \cite{rehman2012off}.

\subsubsection{Аналитический Подход}

Segmentation Based. Классический подход, который используется в распознавании машинопечатных слов, заключается в использовании интеллектуальной системы сегментации, основываясь на свойствах символов, над результатами которой далее работают стандартные методы OCR. Интеллектуальные системы сегментации базируются на эвристических правилах основанными на интуиции исследователей. До сих пор открыта задача разработки автоматической процедуры, которая могла бы обучиться правилам сегментации, используя обучающую выборку, и автоматическому выводом параметров, которые помогли бы в сегментации на достоверных символы. \cite{bunke2003recognition}

\subsubsection{Непрерывный Подход}

Segmentation-Free, Recognition-based Segmentation, HMM-based. В данном подходе сегментация является продуктом системы распознавания. Сегментация возможна, если у распознавателя достаточно информации для поиска наилучшей сегментации. При таком подходе, используются составные модели, в которых модель-слова состоит из конкатенации моделей-символов и соединяющих-моделей. 


\section{Извлечение признаков}

От способа сегментации зависят дальнейшие действия системы: извлечение признаков и распознавание.

\subsection{Подходы}
\subsubsection{Целостный подход}
В целостном подходе к распознаванию используются глобальные признаки слова и поэтому проблема сегментации отсутствует. Слово разделяется на равные фрагменты, далее из каждого фрагмента выделяются признаки, описанные выше. Распознавание слов происходит посредством сравнения его со словами из обучающей выборки, сравнивая схожесть или расстояние между векторами признаков слова и слов из обучающего словаря. Алгоритм К-ближайших соседей может быть использован для решения этой задачи. В литературе также присутствуют работы с применением HMM \cite{lavrenko2004holistic}.

Целостный подход показывает хорошие результаты на маленьких словарях, если вспомнить задачу распознавания банковских чеков, то там размер словаря достаточно мал и следовательно схожих слов по глобальным признакам - мало \cite{guillevic1995unconstrained} . Как только словарь задачи начинает превосходит объединения глоссариев нескольких сфер - ошибка в распознавании сильно увеличивается \cite{rehman2012off}. Эта проблема называется проблемой размерности словаря.

Данный метод не подходит для общей задачи распознавания рукописного текста, но он может быть использован в системах с другими подходами к сегментации/распознаванию. С помощью целостного подхода можно выделять подгруппы из обучающего словаря со схожими глобальными признаками и задавать им больший вес, чем у отсальных слов.

\subsubsection{Аналитический подход}

Аналитический подход отличается от целостного тем, что в аналитический подход основан на нескольких уровнях выделения признаков: уровень букв(фрагментов букв), уровень частей слова, уровень слова.

\subsubsection{Непрерывный подход}

sliding window principle for online offline. Markov models are used for both online and offline handwriting recognition. Especially relevant for the lat- ter, the application of the sliding window principle can be understood as an important ’milestone’ for success- ful Markov-model-based handwriting recognition [...].\cite{plotz2009markov}

\subsection{Признаки скользящего окна}

Описание признаков возможных признаков можно найти в \cite{vinciarelli2003offline} в разделе 2.3.3 Feature Extraction и в \cite{tay2002offline} раздел Handcrafted features

\subsection{Геометричкие признаки}
\subsection{Зонные признаки}
\subsection{Выбор признаков}
\subsubsection{Principal Component Analysis}
\subsubsection{Independent Component Analysis}
\subsubsection{Kernel PCA}

\section{Распознавание марковскими моделями}

Хорошее описание применения динамического программирования к задаче HWR можно найти в \cite{kim1997lexicon} в разделе 6 Recognition

\subsection{Распознавание}
\subsubsection{Алгоритм прямого-обратного хода}
\subsubsection{Алгоритм Витерби}
\subsection{Обучение}
\cite{tay2002offline}

\subsubsection{Алгоритм Баума — Велша}
\subsubsection{Алгоритм min-cut/max-flow}
\subsubsection{Восстановление матрицы эмиссий Гауссовской смесью}
\subsubsection{Восстановление матрицы эмиссий многослойным перцептроном}
\subsection{MMI критерий}
\subsection{ML критерий}

\section{Распознавание нейронными сетями}
\cite{Panwar}
\subsection{Метод обратного распространения ошибки}
\subsection{Long Short-Term Memory}
\subsection{Bidirectional Long Short-Term Memory}
\subsection{Слой CTC}

\section{Адаптация к тексту автора}

\section{Анализ ошибок}



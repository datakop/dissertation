\chapter{Распознавание Рукописного Текста} \label{chapt2}

\section{Мотивация} \label{sect2_1}

Из последних результатов, полученных на сегодняшний день - лидируют следующие работы:
\cite{graves2009novel}

\cite{espana2011improving, zamora2014neural}

\cite{kozielski2013improvements, frinken2014novel, kozielski2014multilingual, pham2014dropout}

\cite{frinken2011co}


\section{Задачи} \label{sect2_2}


\textbf{Handwritten Character Recognition} - одна из сильно изученных проблем распознавания образов. Среди всех подходов к ее решению, ANN самый успешный - [60]. Однако, в распознавании рукописного текста, задача состоит не только в распознавании отдельных рукописных букв, но в отделении этих букв от их соседей, этот процесс называется Сегментацией.
Лидером решения этой задачи является Yann Lecun

\textbf{Handwritten Word Recognition.} В идеальном случае, распознавание рукописных слов может быть сведено к расширению метода распозн букв с применением предварительной сегментации для выделения букв/символов, но данный метод не показал хороших результатов. Сегментация слова не может быть точной без знания самого слова, это так называемый парадокс Sayre-а [62].  Распознаватель бук не может правильно распознать слово без правильной сегментации. Это ведет к принципу использования систем с лексическим движком, в которых сегментация и распознавание тесно сопряжены.

\textbf{Handwritten Sentence Recognition.} \cite{frinken2014novel} Современный подход реализации систем распознавания рукописного текста, на вход которой подается вся строка целиком. В основе этих систем лежит HMM-модель-буквы.


\section{Проблемы} \label{sect2_3}

Проблемы и сложности в задаче распознавания рукописного текста могут быть кратко разделены на 5 категорий:

\textbf{Nature of handwriting signal(offline/online).} В зависимости от сценария, существует два типа рукописных сигналов, которые могут быть прочитаны с  устройств ввода: online рукописный текст и offline рукописный текст.

Offline распознавание рукописного текста сталкивается с проблемой распознавания текста, который был написан до его оцифровки системами сканирования. Данный сигнал представляет собой 2-D множество binary, gray-scale или RGB пикселей.

Online распознавание рукописного текста сталкивается с проблемой распознавания текста в момент его ввода в систему с помощью специального графического планшета. Сигнал представляет собой временной ряд с координатами {x(t), y(t)} траектории пера, полученные в момент времени t. Используя этот сигнал, система может получить взаимное расположение каждой точки, скорость пера по всей траектории, и было ли соприкосновение пера и планшета в конкретный момент времени.

Так как online сигнал состоит из координат движения пера, то по ним легко восстанавливается offline сигнал. А вот сделать обратное - сложная задача, а иногда и невозможная, пример исследования на эту тему можно посмотреть в \cite{поцепаев2004восстановление}. 

В online сигнале хранится больше информации чем в offline, тем самым, добиться более высокой точности распознавания в online распознавании текста проще чем в offline. 

\textbf{Handwriting style}

\textbf{Writer Dependency}

\textbf{Size of Vocabulary}

\textbf{Language} Для зраличных языков используются различные методы, большинство работ исследовано в направлении Latin, Arabic, Brahmic and Non-alphabetic writing systems.

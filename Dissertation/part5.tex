\chapter{Распознавание рукописного текста} \label{chapt5}

Входными данными для системы распознавания рукописных слов являются нормализованные изображения слов, а выходными - слова в кодировке понятной компьютеру.

В литературе используется несколько направлений к распознаванию, метод распознавания напрямую влияет на метод сегментации слова. 

\section{Методы сегментации для HWR} \label{sect5_1}

В \cite{casey1996survey} все методы сегментации разделены на три основных подхода:


\subsection{Holistic(Целостный)} \label{subsect5_1_1}

Этот подход использует слово целиком, без сегментации его на части. Для распознавания используются следующие признаки: кол-во вертикальных линий, длинна, относительное расстояние между нижними штрихами и верхними штрихами, кол-во тех и друх штрихов, замкнутые области(циклы). Данные подход показывает хорошие разультаты на наборах данных с маленьким словарем \cite{rehman2012off}.


\subsection{Segmentation Based(Классический аналитический подход)} \label{subsect5_1_2}

Классический подход, который используется в распознавании машинопечатных слов, заключается в использовании интеллектуальной системы сегментации, основываясь на свойствах символов, над результатами которой далее работают стандартные методы OCR. Интеллектуальные системы сегментации базируются на эвристических правилах основанными на интуиции исследователей. До сих пор открыта задача разработки автоматической процедуры, которая могла бы обучиться правилам сегментации, используя обучающую выборку, и автоматическому выводом параметров, которые помогли бы в сегментации на достоверных символы. \cite{bunke2003recognition}


\subsection{Segmentation-Free, Recognition-based Segmentation, HMM-based} \label{subsect5_1_2}

В данном подходе сегментация является продуктом системы распознавания. Сегментация возможна, если у распознавателя достаточно информации для поиска наилучшей сегментации. При таком подходе, используются составные модели, в которых модель-слова состоит из конкатенации моделей-символов и соединяющих-моделей. 


\section{Feature Extraction} \label{sect5_2}

От способа сегментации зависят дальнейшие действия системы: извлечение признаков и распознавание.


\subsection{Holistic approach to recognition} \label{subsect5_2_1}

В целостном подходе к распознаванию используются глобальные признаки слова и поэтому проблема сегментации отсутствует. Слово разделяется на равные фрагменты, далее из каждого фрагмента выделяются признаки, описанные выше. Распознавание слов происходит посредством сравнения его со словами из обучающей выборки, сравнивая схожесть или расстояние между векторами признаков слова и слов из обучающего словаря. Алгоритм К-ближайших соседей может быть использован для решения этой задачи. В литературе также присутствуют работы с применением HMM \cite{lavrenko2004holistic}.

Целостный подход показывает хорошие результаты на маленьких словарях, если вспомнить задачу распознавания банковских чеков, то там размер словаря достаточно мал и следовательно схожих слов по глобальным признакам - мало \cite{guillevic1995unconstrained} . Как только словарь задачи начинает превосходит объединения глоссариев нескольких сфер - ошибка в распознавании сильно увеличивается \cite{rehman2012off}. Эта проблема называется проблемой размерности словаря.

Данный метод не подходит для общей задачи распознавания рукописного текста, но он может быть использован в системах с другими подходами к сегментации/распознаванию. С помощью целостного подхода можно выделять подгруппы из обучающего словаря со схожими глобальными признаками и задавать им больший вес, чем у отсальных слов.


\subsection{Analytical approach} \label{subsect5_2_2}

Хорошее описание применения динамического программирования к задаче HWR можно найти в \cite{kim1997lexicon} в разделе 6 Recognition
Описание признаков возможных признаков можно найти в \cite{vinciarelli2003offline} в разделе 2.3.3 Feature Extraction
Аналитический подход отличается от целостного тем, что в аналитический подход основан на нескольких уровнях выделения признаков: уровень букв(фрагментов букв), уровень частей слова, уровень слова.


\subsection{скользящее окно} \label{subsect5_2_3}

sliding window principle for online offline. Markov models are used for both online and offline handwriting recognition. Especially relevant for the lat- ter, the application of the sliding window principle can be understood as an important ’milestone’ for success- ful Markov-model-based handwriting recognition [...].\cite{plotz2009markov}



